\documentclass[a4paper]{article}

\usepackage[top=1in, bottom=1in, left=1.25in, right=1.25in]{geometry}
\usepackage{titlesec}
\usepackage{ctex}
\usepackage[linktoc=all,breaklinks=true,urlcolor=magenta,colorlinks=true,bookmarksnumbered=true]{hyperref}
\usepackage{booktabs} % For formal tables
\usepackage{epsfig}
\usepackage{graphicx}
\usepackage{amsmath}
\usepackage{amssymb}
\usepackage{bm}
\usepackage{algorithm}
\usepackage{algorithmic}
\usepackage{url}
\usepackage{subfigure}
\usepackage{bbding}
\usepackage{multirow}
\usepackage{enumitem}
\usepackage{multicol}
\usepackage{listings}
\usepackage{color}
\newcommand*{\Scale}[2][4]{\scalebox{#1}{$#2$}}%

\makeatletter % `@' now normal "letter"
\@addtoreset{equation}{section}
\makeatother  % `@' is restored as "non-letter"
\renewcommand\theequation{\oldstylenums{\thesection}%
  .\oldstylenums{\arabic{equation}}}

%-----------------------------------------BEGIN DOC----------------------------------------

\begin{document}

\newpagestyle{main}{            
  \sethead{}{迁移学习简明手册}{}     %设置页眉
  \setfoot{}{\thepage}{}      %设置页脚,可以在页脚添加 \thepage  显示页数
  \headrule                                      % 添加页眉的下划线c
}
\pagestyle{main}    %使用该style

\renewcommand{\refname}{参考文献} 
\renewcommand{\figurename}{图}
\renewcommand{\tablename}{表}
\renewcommand{\contentsname}{目录}
\renewcommand{\today}{\number\year 年 \number\month 月 \number\day 日}


\title{{\Huge 迁移学习简明手册{\large\linebreak\\}}{\Large 一点心得体会\\版本号:v1.0\linebreak\linebreak
}}
\author{\\
  王晋东\\中国科学院计算技术研究所\\\href{http://tutorial.transferlearning.xyz}{tutorial.transferlearning.xyz}}
\date{2018年4月}
\maketitle
\thispagestyle{empty}

\newpage

%-----------------------------------------ABSTRACT-------------------------------------
\thispagestyle{empty}
\begin{center}
{\Large\bf{摘\ 要\\}}
\end{center}

迁移学习作为机器学习的一大分支,已经取得了长足的进步。本手册简明地介绍迁移学习的概念与基本方法,并对其中的领域自适应问题中的若干代表性方法进行讲述。最后简要探讨迁移学习未来可能的方向。

本手册编写的目的是帮助迁移学习领域的初学者快速入门并掌握基本方法,为自己的研究和应用工作打下良好基础。

本手册的编写逻辑很简单:是什么——介绍迁移学习;为什么——为什么要用迁移学习、为什么能用;怎么办——如何进行迁移(迁移学习方法)。其中,是什么和为什么解决概念问题,这是一切的前提;怎么办是我们的重点,也占据了最多的篇幅。为了最大限度地方便初学者,我们还特别编写了一章上手实践,直接分享实现代码和心得体会。
\newpage

\thispagestyle{empty}
\section*{推荐语}

\textit{看了王晋东同学的“迁移学习小册子”, 点三个赞!迁移学习被认为是机器学习的下一个爆点,但介绍迁移学习的文章却很有限。 这个册子深入浅出,既回顾了迁移学习的发展历史,又囊括了迁移学习的最新进展。 语言流畅,简明通透。 应该对机器学习的入门和提高都有很大帮助!}

——杨强 (迁移学习权威学者,香港科技大学教授,IJCAI president, AAAI/ACM fellow)

\newpage
%-----------------------------------------ABSTRACT-------------------------------------
%-----------------------------------------CONTENT-------------------------------------
\pagenumbering{Roman}
\setcounter{page}{1}
\begingroup           
\begin{multicols}{2}[
	\setlength{\columnseprule}{.4pt}
	\setlength{\columnsep}{18pt}]
	\tableofcontents
\end{multicols}
\endgroup
\newpage

%------------------------------------------TEXT--------------------------------------------

%----------------------------------------OVERVIEW-----------------------------------------
\newpage

\pagenumbering{Roman}
\setcounter{page}{1}

\section*{写在前面}
\addcontentsline{toc}{section}{写在前面}

一直以来都有这样的愿望:无论学习什么知识,总是希望可以快速准确地找到对应的有价值资源进行学习。我相信我们每个人都梦寐以求。然而,越来越多的学科,尤其是我目前从事的计算机科学、人工智能领域,当下正在飞速地发展着。太多的新知识都难以事半功倍地找到快速入手的教程。庄子曰:\textit{“吾生也有涯,而知也无涯。以有涯随无涯,殆已。”}

我只是迁移学习领域一个很普通的博士生,也同样经历了由“一问三不知”到“稍稍理解”的艰难过程。我在2016年初入门迁移学习之时,迁移学习这个概念还未曾像今天一样炙手可热。当时所能找到的学习资源只有两种:别人已发表的论文和已做过的演讲。这些还是不够简单、不够直观。我需要从如此众多的材料中不断归纳,才能站在博士研究的那个圈子的边缘,以便将来可以做出一点点贡献,往圆圈外突破一点点。

相信不只是我,任何一个刚刚入门的学习者都会经历此过程。

\textit{“沉舟侧畔千帆过,病树前头万木春。”}

已所不欲,勿施于人。正是因为我在初学之时也经历过如此沮丧的时期,我才在Github上对迁移学习进行了整理归纳,在知乎网上以“\textit{王晋东不在家}”为名分享自己对于迁移学习和机器学习的理解和教训、在线上线下与大家讨论相关的问题。很欣慰的是,这些免费开放的资源或多或少地,帮助到了一些初学者,使他们更快速地步入迁移学习之门。

但这些还是不太够。Github上的资源模式已经固定,目前主要是进行日常更新,不断加入新的论文和代码。目前还是缺乏一个人人都能上手的初学者教程。也只一次,有读者提问有没有相关的入门教程,能真正从0到1帮助初学者进行入门。

最近,南京大学博士(现任旷视科技南京研究院负责人)魏秀参学长写了一本《解析卷积神经网络—深度学习实践手册》,给很多深度学习的初学者提供了帮助。受他的启发,我也决定将自己在迁移学习领域的一些学习心得体会整理成一本手册,免费进行分享。希望能借此方式,帮助更多的初学者。\textit{我们不谈风月,只谈干货。}

我不是大佬,我也是迁移学习路上的一名小学生。迁移学习领域比我做的好的同龄人太多了。因此,不敢谈什么\textit{指导}。所有的目的都仅为\textit{分享}。

本手册在互联网上免费开放。随着作者理解的深入(以及其他有意者的增补),本手册肯定会不断修改、越来越好。因此,我打算效仿软件的开发、采取版本更新的方式进行管理。

希望未来可以有更多的有志之士加入,让我们的教程日渐丰富。

\newpage

\section*{致谢}
\addcontentsline{toc}{section}{致谢}

本手册编写过程中得到了许多人的帮助。在此对他们表示感谢。

感谢我的导师、中国科学院计算技术研究所的陈益强研究员。是他一直以来保持着对我的信心,相信我能做出好的研究成果,不断鼓励我,经常与我讨论以明确问题,才有了今天的我。陈老师给我提供了优良的实验环境。我一定会更加努力地科研,做出更多更好的研究成果。

感谢香港科技大学计算机系的杨强教授。杨教授作为迁移学习领域国际泰斗,经常不厌其烦地回答我一些研究上的问题。能够得到杨教授的指导,是我的幸运。希望我能在杨教授带领下,做出更踏实的研究成果。

感谢新加坡南洋理工大学的于涵老师。作为我论文的共同作者,于老师认真的写作态度、对论文的把控能力是我一直学习的榜样。于老师还经常鼓励我,希望可以和于老师有着更多合作,发表更好的文章。

感谢清华大学龙明盛助理教授。龙老师在迁移学习领域发表了众多高质量的研究成果,是我入门时学习的榜样。龙老师还经常对我的研究给予指导。希望有机会可以真正和龙老师合作。

感谢美国伊利诺伊大学芝加哥分校的Philip S. Yu教授对我的指导和鼓励。

感谢新加坡A*STAR的郝书吉老师。我博士生涯的发表的第一篇论文是和郝老师合作完成的。正是有了第一篇论文被发表,才增强了我的自信,在接下来的研究中放平心态。

感谢我的好基友、西安电子科技大学博士生段然同学和我的同病相怜,让我们可以一起吐槽读博生活。

感谢我的室友沈建飞、以及实验室同学的支持。

感谢我的知乎粉丝和所有交流过迁移学习的学者对我的支持。

最后感谢我的女友和父母对我的支持。

本手册中出现的插图,绝大多数来源于相应论文的配图。感谢这些作者做出的优秀的研究成果。希望我能早日作出可以比肩的研究。

\newpage
\section*{说明}
\addcontentsline{toc}{section}{手册说明}
本手册的编写目的是帮助迁移学习领域的初学者快速进行入门。我们尽可能绕开那些非常理论的概念,只讲经验方法。我们还配有多方面的代码、数据、论文资料,最大限度地方便初学者。

本手册的方法部分,关注点是近年来持续走热的领域自适应(Domain Adaptation)问题。迁移学习还有其他众多的研究领域。由于作者研究兴趣所在和能力所限,对其他部分的研究只是粗略介绍。非常欢迎从事其他领域研究的读者提供内容。

本手册的每一章节都是\textit{自包含}的,因此,初学者不必从头开始阅读每一部分。直接阅读自己需要的或者自己感兴趣的部分即可。本手册每一章节的信息如下:

第1章介绍了迁移学习的概念,重点解决什么是迁移学习、为什么要进行迁移学习这两个问题。

第2章介绍了迁移学习的研究领域。

第3章介绍了迁移学习的应用领域。

第4章是迁移学习领域的一些基本知识,包括问题定义,域和任务的表示,以及迁移学习的总体思路。特别地,我们提供了较为全面的度量准则介绍。度量准则是迁移学习领域重要的工具。

第5章简要介绍了迁移学习的四种基本方法,即基于样本迁移、基于特征迁移、基于模型迁移、基于关系迁移。

第6章到第8章,介绍了领域自适应的3大类基本的方法,分别是:数据分布自适应法、特征选择法、子空间学习法。

第9章重点介绍了目前持续最火的深度迁移学习方法。

第10章提供了简单的上手实践教程。

第11章对迁移学习进行了展望,提出了未来几个可能的研究方向。

第12章是对全手册的总结。

第13章是附录,提供了迁移学习领域相关的学习资源,以供读者参考。

\textit{\\由于作者水平有限,不足和错误之处,敬请不吝批评指正。}

\textbf{手册的相关资源:}

网站(内含勘误表):\url{http://t.cn/RmasEFe}

开发维护地址: \url{http://github.com/jindongwang/transferlearning-tutorial}

作者的联系方式:

\textit{邮箱}: {\ttfamily jindongwang@outlook.com},\textit{知乎}:{\ttfamily 王晋东不在家}。

\textit{微博}:{\ttfamily 秦汉日记},\textit{个人网站}:\url{http://jd92.wang}。

\input{chaps/ch01_introduction}
\newpage
\section{迁移学习的研究领域} %--------c----------------------

依据目前较流行的机器学习分类方法,机器学习主要可以分为有监督、半监督和无监督机器学习三大类。同理,迁移学习也可以进行这样的分类。需要注意的是,依据的分类准则不同,分类结果也不同。在这一点上,并没有一个统一的说法。我们在这里仅根据目前较流行的方法,对迁移学习的研究领域进行一个大致的划分。

图~\ref{fig-area}给出了迁移学习的常用分类方法总结。

\begin{figure}[htbp]
	\centering
	\includegraphics[scale=0.6]{./figures/fig-area.pdf}
	\caption{迁移学习的研究领域与研究方法分类}
	\label{fig-area}
\end{figure}

大体上讲,迁移学习的分类可以按照四个准则进行:\textit{按目标域有无标签分、按学习方法分、按特征分、按离线与在线形式分}。不同的分类方式对应着不同的专业名词。当然,即使是一个分类下的研究领域,也可能同时处于另一个分类下。下面我们对这些分类方法及相应的领域作简单描述。

\subsection{按目标域标签分}
这种分类方式最为直观。类比机器学习,按照目标领域有无标签,迁移学习可以分为以下三个大类:

\begin{enumerate}
	\item 监督迁移学习 (Supervised Transfer Learning)
	\item 半监督迁移学习 (Semi-Supervised Transfer Learning)
	\item 无监督迁移学习 (Unsupervised Transfer Learning)
\end{enumerate}

显然,少标签或无标签的问题(半监督和无监督迁移学习),是研究的热点和难点。这也是本手册重点关注的领域。

\subsection{按学习方法分类}
按学习方法的分类形式,最早在迁移学习领域的权威综述文章~\cite{pan2010survey}给出定义。它将迁移学习方法分为以下四个大类:

\begin{enumerate}
	\item 基于样本的迁移学习方法 (Instance based Transfer Learning)
	\item 基于特征的迁移学习方法 (Feature based Transfer Learning)
	\item 基于模型的迁移学习方法 (Model based Transfer Learning)
	\item 基于关系的迁移学习方法 (Relation based Transfer Learning)
\end{enumerate}

这是一个很直观的分类方式,按照数据、特征、模型的机器学习逻辑进行区分,再加上不属于这三者中的关系模式。

基于实例的迁移,简单来说就是通过权重重用,对源域和目标域的样例进行迁移。就是说直接对不同的样本赋予不同权重,比如说相似的样本,我就给它高权重,这样我就完成了迁移,非常简单非常非常直接。

基于特征的迁移,就是更进一步对特征进行变换。意思是说,假设源域和目标域的特征原来不在一个空间,或者说它们在原来那个空间上不相似,那我们就想办法把它们变换到一个空间里面,那这些特征不就相似了?这个思路也非常直接。这个方法是用得非常多的,一直在研究,目前是感觉是研究最热的。

基于模型的迁移,就是说构建参数共享的模型。这个主要就是在神经网络里面用的特别多,因为神经网络的结构可以直接进行迁移。比如说神经网络最经典的finetune就是模型参数迁移的很好的体现。

基于关系的迁移,这个方法用的比较少,这个主要就是说挖掘和利用关系进行类比迁移。比如老师上课、学生听课就可以类比为公司开会的场景。这个就是一种关系的迁移。

目前最热的就是基于特征还有模型的迁移,然后基于实例的迁移方法和他们结合起来使用。

迁移学习方法是本手册的重点。我们在后续的篇幅中介绍。

\subsection{按特征分类}

按照特征的属性进行分类,也是一种常用的分类方法。这在最近的迁移学习综述~\cite{weiss2016survey}中给出。按照特征属性,迁移学习可以分为两个大类:

\begin{enumerate}
	\item 同构迁移学习 (Homogeneous Transfer Learning)
	\item 异构迁移学习 (Heterogeneous Transfer Learning)
\end{enumerate}

这也是一种很直观的方式:如果特征语义和维度都相同,那么就是同构;反之,如果特征完全不相同,那么就是异构。举个例子来说,不同图片的迁移,就可以认为是同构;而图片到文本的迁移,则是异构的。

\subsection{按离线与在线形式分}

按照离线学习与在线学习的方式,迁移学习还可以被分为:

\begin{enumerate}
	\item 离线迁移学习 (Offline Transfer Learning)
	\item 在线迁移学习 (Online Transfer Learning)
\end{enumerate}

目前,绝大多数的迁移学习方法,都采用了离线方式。即,源域和目标域均是给定的,迁移一次即可。这种方式的缺点是显而易见的:算法无法对新加入的数据进行学习,模型也无法得到更新。与之相对的,是在线的方式。即随着数据的动态加入,迁移学习算法也可以不断地更新。
\input{chaps/ch03_application}
\newpage
\section{基础知识}

本部分介绍迁移学习领域的一些基本知识。我们对迁移学习的问题进行简单的形式化,给出迁移学习的总体思路,并且介绍目前常用的一些度量准则。本部分中出现的所有符号和表示形式,是以后章节的基础。已有相关知识的读者可以直接跳过。

\subsection{迁移学习的问题形式化}

迁移学习的问题形式化,是进行一切研究的前提。在迁移学习中,有两个基本的概念:\textbf{领域(Domain)}和\textbf{任务(Task)}。它们是最基础的概念。定义如下:

\subsubsection{领域}

\textbf{领域(Domain):} 是进行学习的主体。领域主要由两部分构成:\textit{数据}和\textit{生成这些数据的概率分布}。通常我们用花体$\mathcal{D}$来表示一个domain,用大写斜体$P$来表示一个概率分布。

特别地,因为涉及到迁移,所以对应于两个基本的领域:\textbf{源领域(Source Domain)}和\textbf{目标领域(Target Domain)}。这两个概念很好理解。源领域就是有知识、有大量数据标注的领域,是我们要迁移的对象;目标领域就是我们最终要赋予知识、赋予标注的对象。知识从源领域传递到目标领域,就完成了迁移。

领域上的数据,我们通常用小写粗体$\mathbf{x}$来表示,它也是向量的表示形式。例如,$\mathbf{x}_i$就表示第$i$个样本或特征。用大写的黑体$\mathbf{X}$表示一个领域的数据,这是一种矩阵形式。我们用大写花体$\mathcal{X}$来表示数据的特征空间。

通常我们用小写下标$s$和$t$来分别指代两个领域。结合领域的表示方式,则:$\mathcal{D}_s$表示源领域,$\mathcal{D}_t$表示目标领域。

值得注意的是,概率分布$P$通常只是一个逻辑上的概念,即我们认为不同领域有不同的概率分布,却一般不给出(也难以给出)$P$的具体形式。

\subsubsection{任务}

\textbf{任务(Task):} 是学习的目标。任务主要由两部分组成:\textit{标签}和\textit{标签对应的函数}。通常我们用花体$\mathcal{Y}$来表示一个标签空间,用$f(\cdot)$来表示一个学习函数。

相应地,源领域和目标领域的类别空间就可以分别表示为$\mathcal{Y}_s$和$\mathcal{Y}_t$。我们用小写$y_s$和$y_t$分别表示源领域和目标领域的实际类别。

\subsubsection{迁移学习}

有了上面领域和任务的定义,我们就可以对迁移学习进行形式化。

\textbf{迁移学习(Transfer Learning):} 给定一个有标记的源域$\mathcal{D}_s=\{\mathbf{x}_{i},y_{i}\}^n_{i=1}$和一个无标记的目标域$\mathcal{D}_t=\{\mathbf{x}_{j}\}^{n+m}_{j=n+1}$。这两个领域的数据分布$P(\mathbf{x}_s)$和P($\mathbf{x}_t)$不同,即$P(\mathbf{x}_s) \ne P(\mathbf{x}_t)$。迁移学习的目的就是要借助$\mathcal{D}_s$的知识,来学习目标域$\mathcal{D}_t$的知识(标签)。

更进一步,结合我们前面说过的迁移学习研究领域,迁移学习的定义需要进行如下的考虑:

(1) 特征空间的异同,即$\mathcal{X}_s$和$\mathcal{X}_t$是否相等。

(2) 类别空间的异同:即$\mathcal{Y}_s$和$\mathcal{Y}_t$是否相等。

(3) 条件概率分布的异同:即$Q_s(y_s|\mathbf{x}_s)$和$Q_t(y_t|\mathbf{x}_t)$是否相等。

结合上述形式化,我们给出\textbf{领域自适应(Domain Adaptation)}这一热门研究方向的定义:

\textbf{领域自适应(Domain Adaptation):} 给定一个有标记的源域$\mathcal{D}_s=\{\mathbf{x}_{i},y_{i}\}^n_{i=1}$和一个无标记的目标域$\mathcal{D}_t=\{\mathbf{x}_{j}\}^{n+m}_{j=n+1}$,假定它们的特征空间相同,即$\mathcal{X}_s = \mathcal{X}_t$,并且它们的类别空间也相同,即$\mathcal{Y}_s = \mathcal{Y}_t$以及条件概率分布也相同,即$Q_s(y_s|\mathbf{x}_s) = Q_t(y_t|\mathbf{x}_t)$。但是这两个域的边缘分布不同,即$P_s(\mathbf{x}_s) \ne P_t(\mathbf{x}_t)$。迁移学习的目标就是,利用有标记的数据$\mathcal{D}_s$去学习一个分类器$f:\mathbf{x}_t \mapsto \mathbf{y}_t$来预测目标域$\mathcal{D}_t$的标签$\mathbf{y}_t \in \mathcal{Y}_t$.

在实际的研究和应用中,读者可以针对自己的不同任务,结合上述表述,灵活地给出相关的形式化定义。

\textbf{符号小结}

我们已经基本介绍了迁移学习中常用的符号。表~\ref{tb-symbol}是一个符号表:

\begin{table}[htbp]
	\centering
	\caption{迁移学习形式化表示常用符号}
	\label{tb-symbol}
	\begin{tabular}{|c|c|}
		\hline
		\textbf{符号} & \textbf{含义} \\ \hline
		下标$s$ / $t$ & 指示源域 / 目标域 \\ \hline
		$\mathcal{D}_s$ / $\mathcal{D}_t$ & 源域数据 / 目标域数据 \\ \hline
		$\mathbf{x}$ /  $\mathbf{X}$ / $\mathcal{X}$ & 向量 / 矩阵 / 特征空间 \\ \hline
		$\mathbf{y}$ / $\mathcal{Y}$ & 类别向量 / 类别空间 \\ \hline
		$(n,m)$ [或 $(n_1,n_2)$ 或 $(n_s,n_t)$] & (源域样本数,目标域样本数) \\ \hline
		$P(\mathbf{x}_s)$ / $P(\mathbf{x}_t)$ & 源域数据 / 目标域数据的边缘分布 \\ \hline
		$Q(\mathbf{y}_s | \mathbf{x}_s)$ / $Q(\mathbf{y}_t | \mathbf{x}_t)$ & 源域数据 / 目标域数据的条件分布 \\ \hline
		$f(\cdot)$ & 要学习的目标函数 \\ \hline
	\end{tabular}
\end{table}

\subsection{总体思路}

形式化之后,我们可以进行迁移学习的研究。迁移学习的总体思路可以概括为:\textit{开发算法来最大限度地利用有标注的领域的知识,来辅助目标领域的知识获取和学习}。

迁移学习的核心是,找到源领域和目标领域之间的\textbf{相似性},并加以合理利用。这种相似性非常普遍。比如,不同人的身体构造是相似的;自行车和摩托车的骑行方式是相似的;国际象棋和中国象棋是相似的;羽毛球和网球的打球方式是相似的。这种相似性也可以理解为\textbf{不变量}。以不变应万变,才能立于不败之地。

举一个杨强教授经常举的例子来说明:我们都知道在中国大陆开车时,驾驶员坐在左边,靠马路右侧行驶。这是基本的规则。然而,如果在英国、香港等地区开车,驾驶员是坐在右边,需要靠马路左侧行驶。那么,如果我们从中国大陆到了香港,应该如何快速地适应他们的开车方式呢?诀窍就是找到这里的不变量:\textit{不论在哪个地区,驾驶员都是紧靠马路中间。}这就是我们这个开车问题中的不变量。

找到相似性(不变量),是进行迁移学习的核心。

有了这种相似性后,下一步工作就是,\textit{如何度量和利用这种相似性}。度量工作的目标有两点:一是很好地度量两个领域的相似性,不仅定性地告诉我们它们是否相似,更\textit{定量}地给出相似程度。二是以度量为准则,通过我们所要采用的学习手段,增大两个领域之间的相似性,从而完成迁移学习。

\textbf{一句话总结:} \textit{相似性是核心,度量准则是重要手段}。

\subsection{度量准则}

度量不仅是机器学习和统计学等学科中使用的基础手段,也是迁移学习中的重要工具。它的核心就是衡量两个数据域的差异。计算两个向量(点、矩阵)的距离和相似度是许多机器学习算法的基础,有时候一个好的距离度量就能决定算法最后的结果好坏。比如KNN分类算法就对距离非常敏感。本质上就是找一个变换使得源域和目标域的距离最小(相似度最大)。所以,相似度和距离度量在机器学习中非常重要。

这里给出常用的度量手段,它们都是迁移学习研究中非常常见的度量准则。对这些准则有很好的理解,可以帮助我们设计出更加好用的算法。用一个简单的式子来表示,度量就是描述源域和目标域这两个领域的距离:

\begin{equation}
	\label{eq-distance}
	DISTANCE(\mathcal{D}_s,\mathcal{D}_t) = \mathrm{DistanceMeasure}(\cdot,\cdot)
\end{equation}

下面我们从距离和相似度度量准则几个方面进行简要介绍。

\subsubsection{常见的几种距离}

\textbf{1. 欧氏距离}

定义在两个向量(空间中的两个点)上:点$\mathbf{x}$和点$\mathbf{y}$的欧氏距离为:

\begin{equation}
	\label{eq-dist-eculidean}
	d_{Euclidean}=\sqrt{(\mathbf{x}-\mathbf{y})^\top (\mathbf{x}-\mathbf{y})}
\end{equation}

c
\textbf{2. 闵可夫斯基距离} 

Minkowski distance, 两个向量(点)的$p$阶距离:

\begin{equation}
	\label{eq-dist-minkowski}
	d_{Minkowski}=(||\mathbf{x}-\mathbf{y}||^p)^{1/p}
\end{equation}

当$p=1$时就是曼哈顿距离,当$p=2$时就是欧氏距离。

\textbf{3. 马氏距离}

定义在两个向量(两个点)上,这两个数据在同一个分布里。点$\mathbf{x}$和点$\mathbf{y}$的马氏距离为:

\begin{equation}
	\label{eq-dist-maha}
	d_{Mahalanobis}=\sqrt{(\mathbf{x}-\mathbf{y})^\top \Sigma^{-1} (\mathbf{x}-\mathbf{y})}
\end{equation}

其中,$\Sigma$是这个分布的协方差。

当$\Sigma=\mathbf{I}$时,马氏距离退化为欧氏距离。

\subsubsection{相似度}

\textbf{1. 余弦相似度}

衡量两个向量的相关性(夹角的余弦)。向量$\mathbf{x},\mathbf{y}$的余弦相似度为:

\begin{equation}
	\label{eq-dist-cosine}
	\cos (\mathbf{x},\mathbf{y}) = \frac{\mathbf{x} \cdot \mathbf{y}}{|\mathbf{x}|\cdot |\mathbf{y}|}
\end{equation}

\textbf{2. 互信息}

定义在两个概率分布$X,Y$上,$x \in X, y \in Y$。它们的互信息为:

\begin{equation}
	\label{eq-dist-info}
	I(X;Y)=\sum_{x \in X} \sum_{y \in Y} p(x,y) \log \frac{p(x,y)}{p(x)p(y)}
\end{equation}

\textbf{3. 皮尔逊相关系数}

衡量两个随机变量的相关性。随机变量$X,Y$的Pearson相关系数为:

\begin{equation}
	\label{eq-dist-pearson}
	\rho_{X,Y}=\frac{Cov(X,Y)}{\sigma_X \sigma_Y}
\end{equation}

理解:协方差矩阵除以标准差之积。

范围:$[-1,1]$,绝对值越大表示(正/负)相关性越大。

\textbf{4. Jaccard相关系数}

对两个集合$X,Y$,判断他们的相关性,借用集合的手段:

\begin{equation}
	\label{eq-dist-jaccard}
	J=\frac{X \cap Y}{X \cup Y}
\end{equation}

理解:两个集合的交集除以并集。

扩展:Jaccard距离=$1-J$。

\subsubsection{KL散度与JS距离}

KL散度和JS距离是迁移学习中被广泛应用的度量手段。

\textbf{1. KL散度}

Kullback–Leibler divergence,又叫做\textit{相对熵},衡量两个概率分布$P(x),Q(x)$的距离:

\begin{equation}
	\label{eq-dist-kl}
	D_{KL}(P||Q)=\sum_{i=1} P(x) \log \frac{P(x)}{Q(x)}
\end{equation}

这是一个非对称距离:$D_{KL}(P||Q) \ne D_{KL}(Q||P)$.

\textbf{2. JS距离}

Jensen–Shannon divergence,基于KL散度发展而来,是对称度量:

\begin{equation}
	\label{eq-dist-js}
	JSD(P||Q)= \frac{1}{2} D_{KL}(P||M) + \frac{1}{2} D_{KL}(Q||M)
\end{equation}

其中$M=\frac{1}{2}(P+Q)$。

\subsubsection{最大均值差异MMD}

最大均值差异是迁移学习中使用频率最高的度量。Maximum mean discrepancy,它度量在再生希尔伯特空间中两个分布的距离,是一种核学习方法。两个随机变量的MMD平方距离为

\begin{equation}
	\label{eq-dist-mmd}
	MMD^2(X,Y)=\left \Vert \sum_{i=1}^{n_1}\phi(\mathbf{x}_i)- \sum_{j=1}^{n_2}\phi(\mathbf{y}_j) \right \Vert^2_\mathcal{H}
\end{equation}

其中$\phi(\cdot)$是映射,用于把原变量映射到\textit{再生核希尔伯特空间}(Reproducing Kernel Hilbert Space, RKHS)~\cite{borgwardt2006integrating}中。什么是RKHS?形式化定义太复杂,简单来说希尔伯特空间是对于函数的内积完备的,而再生核希尔伯特空间是具有再生性$\langle K(x,\cdot),K(y,\cdot)\rangle_\mathcal{H}=K(x,y)$的希尔伯特空间。就是比欧几里得空间更高端的。将平方展开后,RKHS空间中的内积就可以转换成核函数,所以最终MMD可以直接通过核函数进行计算。

理解:就是求两堆数据在RKHS中的\textit{均值}的距离。

\textit{Multiple-kernel MMD}:多核的MMD,简称MK-MMD。现有的MMD方法是基于单一核变换的,多核的MMD假设最优的核可以由多个核线性组合得到。多核MMD的提出和计算方法在文献~\cite{gretton2012optimal}中形式化给出。MK-MMD在许多后来的方法中被大量使用,最著名的方法是DAN~\cite{long2015learning}。我们将在后续单独介绍此工作。

\subsubsection{Principal Angle}

也是将两个分布映射到高维空间(格拉斯曼流形)中,在流形中两堆数据就可以看成两个点。Principal angle是求这两堆数据的对应维度的夹角之和。

对于两个矩阵$\mathbf{X},\mathbf{Y}$,计算方法:首先正交化(用PCA)两个矩阵,然后:

\begin{equation}
\label{eq-dist-pa}
PA(\mathbf{X},\mathbf{Y})=\sum_{i=1}^{\min(m,n)} \sin \theta_i
\end{equation}

其中$m,n$分别是两个矩阵的维度,$\theta_i$是两个矩阵第$i$个维度的夹角,$\Theta=\{\theta_1,\theta_2,\cdots,\theta_t\}$是两个矩阵SVD后的角度:

\begin{equation}
	\mathbf{X}^\top\mathbf{Y}=\mathbf{U} (\cos \Theta) \mathbf{V}^\top
\end{equation}

\subsubsection{A-distance}

$\mathcal{A}$-distance是一个很简单却很有用的度量。文献\cite{ben2007analysis}介绍了此距离,它可以用来估计不同分布之间的差异性。$\mathcal{A}$-distance被定义为建立一个线性分类器来区分两个数据领域的hinge损失(也就是进行二类分类的hinge损失)。它的计算方式是,我们首先在源域和目标域上训练一个二分类器$h$,使得这个分类器可以区分样本是来自于哪一个领域。我们用$err(h)$来表示分类器的损失,则$\mathcal{A}$-distance定义为:

\begin{equation}
	\label{eq-dist-adist}
	\mathcal{A}(\mathcal{D}_s,\mathcal{D}_t) = 2(1 - 2 err(h))
\end{equation}

$\mathcal{A}$-distance通常被用来计算两个领域数据的相似性程度,以便与实验结果进行验证对比。

\subsubsection{Hilbert-Schmidt Independence Criterion}

希尔伯特-施密特独立性系数,Hilbert-Schmidt Independence Criterion,用来检验两组数据的独立性:
\begin{equation}
	HSIC(X,Y) = trace(HXHY)
\end{equation}
其中$X,Y$是两堆数据的kernel形式。

\subsubsection{Wasserstein Distance}

Wasserstein Distance是一套用来衡量两个概率分部之间距离的度量方法。该距离在一个度量空间$(M,\rho)$上定义,其中$\rho(x,y)$表示集合$M$中两个实例$x$和$y$的距离函数,比如欧几里得距离。两个概率分布$\mathbb{P}$和$\mathbb{Q}$之间的$p{\text{-th}}$ Wasserstein distance可以被定义为

\begin{equation}
W_p(\mathbb{P}, \mathbb{Q}) = \Big(\inf_{\mu \in \Gamma(\mathbb{P}, \mathbb{Q}) } \int \rho(x,y)^p d\mu(x,y) \Big)^{1/p},
\end{equation}

其中$\Gamma(\mathbb{P}, \mathbb{Q})$是在集合$M\times M$内所有的以$\mathbb{P}$和$\mathbb{Q}$为边缘分布的联合分布。著名的Kantorovich-Rubinstein定理表示当$M$是可分离的时候,第一Wasserstein distance可以等价地表示成一个积分概率度量(integral probability metric)的形式

\begin{equation}
W_1(\mathbb{P},\mathbb{Q})= \sup_{\left \| f \right \|_L \leq 1} \mathbb{E}_{x \sim \mathbb{P}}[f(x)] - \mathbb{E}_{x \sim \mathbb{Q}}[f(x)],
\end{equation}
其中$\left \| f \right \|_L = \sup{|f(x) - f(y)|} / \rho(x,y)$并且$\left \| f \right \|_L \leq 1$称为$1-$利普希茨条件。

\subsection{迁移学习的理论保证*}
\textit{
本部分的标题中带有*号,有一些难度,为可看可不看的内容。此部分最常见的形式是当自己提出的算法需要理论证明时,可以借鉴。}

在第一章里我们介绍了两个重要的概念:迁移学习是什么,以及为什么需要迁移学习。但是,还有一个重要的问题没有得到解答:\textit{为什么可以进行迁移}?也就是说,迁移学习的可行性还没有探讨。

值得注意的是,就目前的研究成果来说,迁移学习领域的理论工作非常匮乏。我们在这里仅回答一个问题:为什么数据分布不同的两个领域之间,知识可以进行迁移?或者说,到底达到什么样的误差范围,我们才认为知识可以进行迁移?

加拿大滑铁卢大学的Ben-David等人从2007年开始,连续发表了三篇文章~\cite{ben2007analysis,blitzer2008learning,ben2010theory}对迁移学习的理论进行探讨。在文中,作者将此称之为“Learning from different domains”。在三篇文章也成为了迁移学习理论方面的经典文章。文章主要回答的问题就是:在怎样的误差范围内,从不同领域进行学习是可行的?

\textbf{学习误差:} 给定两个领域$\mathcal{D}_s,\mathcal{D}_t$,$X$是定义在它们之上的数据,一个假设类$\mathcal{H}$。则两个领域$\mathcal{D}_s,\mathcal{D}_t$之间的$\mathcal{H}$-divergence被定义为

\begin{equation}
	\hat{d}_{\mathcal{H}}(\mathcal{D}_s,\mathcal{D}_t) = 2 \sup_{\eta \in \mathcal{H}} \left|\underset{\mathbf{x} \in \mathcal{D}_s}{P}[\eta(\mathbf{x}) = 1] - \underset{\mathbf{x} \in \mathcal{D}_t}{P}[\eta(\mathbf{x}) = 1] \right|
\end{equation}

因此,这个$\mathcal{H}$-divergence依赖于假设$\mathcal{H}$来判别数据是来自于$\mathcal{D}_s$还是$\mathcal{D}_t$。作者证明了,对于一个对称的$\mathcal{H}$,我们可以通过如下的方式进行计算

\begin{equation}
	d_\mathcal{H} (\mathcal{D}_s,\mathcal{D}_t) = 2 \left(1 - \min_{\eta \in \mathcal{H}} \left[\frac{1}{n_1} \sum_{i=1}^{n_1} I[\eta(\mathbf{x}_i)=0] + \frac{1}{n_2} \sum_{i=1}^{n_2} I[\eta(\mathbf{x}_i)=1]\right] \right)
\end{equation}
其中$I[a]$为指示函数:当$a$成立时其值为1,否则其值为0。

\textbf{在目标领域的泛化界:}

假设$\mathcal{H}$为一个具有$d$个VC维的假设类,则对于任意的$\eta \in \mathcal{H}$,下面的不等式有$1 - \delta$的概率成立:

\begin{equation}
	R_{\mathcal{D}_t}(\eta) \le R_s(\eta) + \sqrt{\frac{4}{n}(d \log \frac{2en}{d} + \log \frac{4}{\delta})} + \hat{d}_{\mathcal{H}}(\mathcal{D}_s,\mathcal{D}_t) + 4 \sqrt{\frac{4}{n}(d \log \frac{2n}{d} + \log \frac{4}{\delta})} + \beta
\end{equation}
其中
\begin{equation}
	\beta \ge \inf_{\eta^\star \in \mathcal{H}} [R_{\mathcal{D}_s}(\eta^\star) + R_{\mathcal{D}_t}(\eta^\star)]
\end{equation}
并且
\begin{equation}
	R_{s}(\eta) = \frac{1}{n} \sum_{i=1}^{m} I[\eta(\mathbf{x}_i) \ne y_i]
\end{equation}

具体的理论证明细节,请参照上述提到的三篇文章。

在自己的研究中,如果需要进行相关的证明,可以参考一些已经发表的文章的写法,例如~\cite{long2014adaptation}等。

另外,英国的Gretton等人也在进行一些学习理论方面的研究,有兴趣的读者可以关注他的个人主页:\url{http://www.gatsby.ucl.ac.uk/~gretton/}。

\input{chaps/ch05_method}
\newpage
\section{第一类方法:数据分布自适应}

数据分布自适应(Distribution Adaptation)是一类最常用的迁移学习方法。这种方法的基本思想是,由于源域和目标域的数据概率分布不同,那么最直接的方式就是通过一些变换,将不同的数据分布的距离拉近。

图~\ref{fig-distribution}形象地表示了几种数据分布的情况。\textit{简单来说,数据的边缘分布不同,就是数据整体不相似。数据的条件分布不同,就是数据整体相似,但是具体到每个类里,都不太相似。}

\begin{figure}[htbp]
	\centering
	\subfigure[源域数据]{
		\includegraphics[scale=0.5]{./figures/fig-distribution-source.pdf}
		\label{fig-distribution-source}}
	~\vline
	~
	\subfigure[目标域数据: 类型I]{
		\includegraphics[scale=0.5]{./figures/fig-distribution-target1.pdf}
		\label{fig-distribution-target1}}
	~
	\subfigure[目标域数据:类型II]{
		\includegraphics[scale=0.5]{./figures/fig-distribution-target2.pdf}
		\label{fig-distribution-target2}}
	\caption{不同数据分布的目标域数据}
	\label{fig-distribution}
\end{figure}

根据数据分布的性质,这类方法又可以分为\textit{边缘分布自适应、条件分布自适应}、以及\textit{联合分布自适应}。下面我们分别介绍每类方法的基本原理和代表性研究工作。介绍每类研究工作时,我们首先给出基本思路,然后介绍该类方法的核心,最后结合最近的相关工作介绍该类方法的扩展。

\subsection{边缘分布自适应}

\subsubsection{基本思路}

边缘分布自适应方法(Marginal Distribution Adaptation)的目标是减小源域和目标域的边缘概率分布的距离,从而完成迁移学习。从形式上来说,边缘分布自适应方法是用$P(\mathbf{x}_s)$和$P(\mathbf{x}_t)$之间的距离来近似两个领域之间的差异。即:

\begin{equation}
	\label{eq-marginal-general}
	DISTANCE(\mathcal{D}_s,\mathcal{D}_t) \approx ||P(\mathbf{x}_s) - P(\mathbf{x}_t)||
\end{equation}

边缘分布自适应对应于图~\ref{fig-distribution}中由图~\ref{fig-distribution-source}迁移到图~\ref{fig-distribution-target1}的情形。

\subsubsection{核心方法}

边缘分布自适应的方法最早由香港科技大学杨强教授团队提出~\cite{pan2011domain},方法名称为迁移成分分析(Transfer Component Analysis)。由于$P(\mathbf{x}_s) \ne P(\mathbf{x}_t)$,因此,直接减小二者之间的距离是不可行的。TCA假设存在一个特征映射$\phi$,使得映射后数据的分布$P(\phi(\mathbf{x}_s)) \approx P(\phi(\mathbf{x}_t))$。TCA假设如果边缘分布接近,那么两个领域的条件分布也会接近,即条件分布$P(y_s | \phi(\mathbf{x}_s))) \approx P(y_t | \phi(\mathbf{x}_t)))$。这就是TCA的全部思想。因此,我们现在的目标是,找到这个合适的$\phi$。

但是世界上有无穷个这样的$\phi$,也许终我们一生也无法找到合适的那一个。庄子说过,吾生也有涯,而知也无涯,以有涯随无涯,殆已!我们肯定不能通过穷举的方法来找$\phi$的。那么怎么办呢?

回到迁移学习的本质上来:最小化源域和目标域的距离。好了,我们能不能先假设这个$\phi$是已知的,然后去求距离,看看能推出什么呢?

更进一步,这个距离怎么算?机器学习中有很多种形式的距离,从欧氏距离到马氏距离,从曼哈顿距离到余弦相似度,我们需要什么距离呢?TCA利用了一个经典的也算是比较“高端”的距离叫做最大均值差异(MMD,maximum mean discrepancy)。我们令$n_1,n_2$分别表示源域和目标域的样本个数,那么它们之间的MMD距离可以计算为:

\begin{equation}
	\label{eq-distribution-mmd}
	DISTANCE(\mathbf{x}_{s},\mathbf{x}_{t})= \begin{Vmatrix} \frac{1}{n_1} \sum \limits_{i=1}^{n_1} \phi(\mathbf{x}_{i}) - \frac{1}{n_2}\sum \limits _{j=1}^{n_2} \phi(\mathbf{x}_{j}) \end{Vmatrix}_{\mathcal{H}}
\end{equation}

MMD是做了一件什么事呢?简单,就是求映射后源域和目标域的\textit{均值之差}。

事情到这里似乎也没什么进展:我们想求的$\phi$仍然没法求。

TCA是怎么做的呢,这里就要感谢矩阵了!我们发现,上面这个MMD距离平方展开后,有二次项乘积的部分!那么,联系在SVM中学过的核函数,把一个难求的映射以核函数的形式来求,不就可以了?于是,TCA引入了一个核矩阵$\mathbf{K}$:

\begin{equation}
	\mathbf{K}=\begin{bmatrix}\mathbf{K}_{s,s} & \mathbf{K}_{s,t}\\\mathbf{K}_{t,s} & \mathbf{K}_{t,t}\end{bmatrix} 
\end{equation}

以及一个MMD矩阵$\mathbf{L}$,它的每个元素的计算方式为:

\begin{equation}
	l_{ij}=\begin{cases} \frac{1}{{n_1}^2} & \mathbf{x}_i,\mathbf{x}_j \in \mathcal{D}_s,\\ \frac{1}{{n_2}^2} & \mathbf{x}_i,\mathbf{x}_j \in \mathcal{D}_t,\\ -\frac{1}{n_1 n_2} & \text{otherwise} \end{cases}
\end{equation}

这样的好处是,直接把那个难求的距离,变换成了下面的形式:

\begin{equation}
	\mathrm{tr}(\mathbf{KL})-\lambda \mathrm{tr}(\mathbf{K})
\end{equation}

其中,$\mathrm{tr}(\cdot)$操作表示求矩阵的迹,用人话来说就是一个矩阵对角线元素的和。这样是不是感觉离目标又进了一步呢?

其实这个问题到这里就已经是可解的了,也就是说,属于计算机的部分已经做完了。只不过它是一个数学中的半定规划(SDP,semi-definite programming)的问题,解决起来非常耗费时间。由于TCA的第一作者Sinno Jialin Pan以前是中山大学的数学硕士,他想用更简单的方法来解决。他是怎么做的呢?

他想出了用降维的方法去构造结果。用一个更低维度的矩阵$\mathbf{W}$:
\begin{equation}
	\widetilde{\mathbf{K}}=({\mathbf{K}}{\mathbf{K}}^{-1/2}\widetilde{\mathbf{W}})(\widetilde{\mathbf{W}}^{\top}{\mathbf{K}}^{-1/2}{\mathbf{K}})={\mathbf{K}}\mathbf{W} \mathbf{W}^{\top}{\mathbf{K}}
\end{equation}

这里的$\mathbf{W}$矩阵是比$\mathbf{K}$更低维度的矩阵。最后的$\mathbf{W}$就是问题的解答了!

好了,问题到这里,整理一下,TCA最后的优化目标是:

\begin{equation}
	\begin{split} \min_\mathbf{W} \quad& \mathrm{tr}(\mathbf{W}^\top \mathbf{K} \mathbf{L} \mathbf{K} \mathbf{W}) + \mu \mathrm{tr}(\mathbf{W}^\top \mathbf{W})\\ \text{s.t.} \quad & \mathbf{W}^\top \mathbf{K} \mathbf{H} \mathbf{K} \mathbf{W} = \mathbf{I}_m \end{split} 
\end{equation}

这里的$\mathbf{H}$是一个中心矩阵,$\mathbf{H} = \mathbf{I}_{n_1 + n_2} - 1/(n_1 + n_2)\mathbf{11}^\top$.

这个式子下面的条件是什么意思呢?那个$\min$的目标我们大概理解,就是要最小化源域和目标域的距离,加上$\mathbf{W}$的约束让它不能太复杂。那么下面的条件是什么呢?下面的条件就是要实现第二个目标:维持各自的数据特征。

TCA要维持的是什么特征呢?文章中说是variance,但是实际是scatter matrix,就是数据的散度。就是说,一个矩阵散度怎么计算?对于一个矩阵$\mathbf{A}$,它的scatter matrix就是$\mathbf{A} \mathbf{H} \mathbf{A}^\top$。这个$\mathbf{H}$就是上面的中心矩阵啦。

解决上面的优化问题时,作者又求了它的拉格朗日对偶。最后得出结论,$\mathbf{W}$的解就是它的前$m$个特征值!简单不?数学美不美?

好了,我们现在总结一下TCA方法的步骤。输入是两个特征矩阵,我们首先计算$\mathbf{L}$和$\mathbf{H}$矩阵,然后选择一些常用的核函数进行映射(比如线性核、高斯核)计算$\mathbf{K}$,接着求$({\mathbf{K}} \mathbf{L} {\mathbf{K}}+\mu \mathbf{I})^{-1}{\mathbf{K}} \mathbf{H}{\mathbf{K}}$的前$m$个特征值。仅此而已。然后,得到的就是源域和目标域的降维后的数据,我们就可以在上面用传统机器学习方法了。

为了形象地展示TCA方法的优势,我们借用~\cite{pan2011domain}中提供的可视化效果,在图中展示了对于源域和目标域数据(红色和蓝色),分别由PCA(主成分分析)和TCA得到的分布结果。从图~\ref{fig-distribution-tca}中可以很明显地看出,对于概率分布不同的两部分数据,在经过TCA处理后,概率分布更加接近。这说明了TCA在拉近数据分布距离上的优势。

\begin{figure}[htbp]
	\centering
	\includegraphics[scale=0.6]{./figures/fig-distribution-tca.pdf}
	\caption{TCA和PCA的效果对比}
	\label{fig-distribution-tca}
\end{figure}

\subsubsection{扩展}

TCA方法是迁移学习领域一个经典的方法,之后的许多研究工作都以TCA为基础。我们列举部分如下:

\begin{itemize}
	\item ACA (Adapting Component Analysis)~\cite{dorri2012adapting}: 在TCA中加入HSIC
	\item DTMKL (Domain Transfer Multiple Kernel Learning)~\cite{duan2012domain}: 在TCA中加入了MK-MMD,用了新的求解方式
	\item TJM (Transfer Joint Matching)~\cite{long2014transfer}: 在优化目标中同时进行边缘分布自适应和源域样本选择
	\item DDC (Deep Domain Confusion)~\cite{tzeng2014deep}: 将MMD度量加入了深度网络特征层的loss中(我们将会在深度迁移学习中介绍此工作)
	\item DAN (Deep Adaptation Network)~\cite{long2015learning}: 扩展了DDC的工作,将MMD换成了MK-MMD,并且进行多层loss计算(我们将会在深度迁移学习中介绍此工作)
	\item DME (Distribution Matching Embedding): 先计算变换矩阵,再进行特征映射(与TCA顺序相反)
	\item CMD (Central Moment Matching)~\cite{zellinger2017central}: MMD着眼于一阶,此工作将MMD推广到了多阶
\end{itemize}

\subsection{条件分布自适应}

条件分布自适应方法(Conditional Distribution Adaptation)的目标是减小源域和目标域的条件概率分布的距离,从而完成迁移学习。从形式上来说,条件分布自适应方法是用$P(y_s|\mathbf{x}_s)$和$P(y_t|\mathbf{x}_t)$之间的距离来近似两个领域之间的差异。即:

\begin{equation}
\label{eq-conditional-general}
DISTANCE(\mathcal{D}_s,\mathcal{D}_t) \approx ||P(y_s|\mathbf{x}_s) - P(y_t|\mathbf{x}_t)||
\end{equation}

条件分布自适应对应于图~\ref{fig-distribution}中由图~\ref{fig-distribution-source}迁移到图~\ref{fig-distribution-target2}的情形。

目前单独利用条件分布自适应的工作较少,这些工作主要可以在~\cite{saito2017asymmetric}中找到。最近,中科院计算所的Wang等人提出了STL方法(Stratified Transfer Learning)~\cite{wang2018stratified}。作者提出了\textit{类内迁移}(Intra-class Transfer)的思想。指出现有的绝大多数方法都只是学习一个全局的特征变换(Global Domain Shift),而忽略了类内的相似性。类内迁移可以利用类内特征,实现更好的迁移效果。

STL方法的基本思路如图~\ref{fig-distribution-stl}所示。首先利用大多数投票的思想,对无标定的位置行为生成伪标签;然后在再生核希尔伯特空间中,利用类内相关性进行自适应地空间降维,使得不同情境中的行为数据之间的相关性增大;最后,通过二次标定,实现对未知标定数据的精准标定。

\begin{figure}[htbp]
	\centering
	\includegraphics[scale=0.85]{./figures/fig-distribution-stl.pdf}
	\caption{STL方法的示意图}
	\label{fig-distribution-stl}
\end{figure}

为了实现\textit{类内迁移},我们需要计算每一类别的MMD距离。由于目标域没有标记,作者使用来自大多数投票结果中的伪标记。更加准确地说,用$c \in \{1, 2, \cdots, C\}$来表示类别标记,则类内迁移可以按如下方式计算:

\begin{equation}
\label{equ-stl-stra}
D(\mathcal{D}_{s},\mathcal{D}_{t})
=\sum_{c=1}^{C}\left \Vert \frac{1}{n^{(c)}_1} \sum_{\mathbf{x}_i \in \mathcal{D}^{(c)}_s} \phi(\mathbf{x}_i) - \frac{1}{n^{(c)}_2} \sum_{\mathbf{x}_j \in \mathcal{D}^{(c)}_t} \phi(\mathbf{x}_j) \right \Vert ^2_\mathcal{H}
\end{equation}

其中,$\mathcal{D}^{(c)}_s$和$\mathcal{D}^{(c)}_t$分别表示源域和目标域中属于类别$c$的样本。$n^{(c)}_1=|\mathcal{D}^{(c)}_s|$,且$n^{(c)}_2=|\mathcal{D}_t|$。

接下来的步骤请参照STL方法原文进行理解。

STL方法在大量行为识别数据中进行了跨位置行为识别的实验。实验结果表明,该方法可以很好地实现跨领域的行为识别任务,取得了当前最好的效果。

\subsection{联合分布自适应}

\subsubsection{基本思路}

联合分布自适应方法(Joint Distribution Adaptation)的目标是减小源域和目标域的联合概率分布的距离,从而完成迁移学习。从形式上来说,联合分布自适应方法是用$P(\mathbf{x}_s)$和$P(\mathbf{x}_t)$之间的距离、以及$P(y_s|\mathbf{x}_s)$和$P(y_t|\mathbf{x}_t)$之间的距离来近似两个领域之间的差异。即:

\begin{equation}
\label{eq-joint-general}
DISTANCE(\mathcal{D}_s,\mathcal{D}_t) \approx ||P(\mathbf{x}_s) - P(\mathbf{x}_t)|| + ||P(y_s|\mathbf{x}_s) - P(y_t|\mathbf{x}_t)||
\end{equation}

联合分布自适应对应于图~\ref{fig-distribution}中由图~\ref{fig-distribution-source}迁移到图~\ref{fig-distribution-target1}的情形、以及图~\ref{fig-distribution-source}迁移到图~\ref{fig-distribution-target2}的情形。

\subsubsection{核心方法}

联合分布适配的JDA方法~\cite{long2013transfer}首次发表于2013年的ICCV(计算机视觉领域顶会,与CVPR类似),它的作者是当时清华大学的博士生(现为清华大学助理教授)龙明盛。

假设是最基本的出发点。那么JDA这个方法的假设是什么呢?就是假设两点:1)源域和目标域边缘分布不同,2)源域和目标域条件分布不同。既然有了目标,同时适配两个分布不就可以了吗?于是作者很自然地提出了联合分布适配方法:适配联合概率。

不过这里我感觉有一些争议:边缘分布和条件分布不同,与联合分布不同并不等价。所以这里的“联合”二字实在是会引起歧义。我的理解是,同时适配两个分布,也可以叫联合,而不是概率上的“联合”。尽管作者在文章里第一个公式就写的是适配联合概率,但是这里感觉是有一些问题的。我们抛开它这个有歧义的,把“联合”理解成同时适配两个分布。

那么,JDA方法的目标就是,寻找一个变换$\mathbf{A}$,使得经过变换后的$P(\mathbf{A}^\top \mathbf{x}_s)$和$P(\mathbf{A}^\top \mathbf{x}_t)$的距离能够尽可能地接近,同时,$P(y_s|\mathbf{A}^\top \mathbf{x}_s)$和$P(y_t|\mathbf{A}^\top \mathbf{x}_t)$的距离也要小。很自然地,这个方法也就分成了两个步骤。

\textit{边缘分布适配}

首先来适配边缘分布,也就是$P(\mathbf{A}^\top \mathbf{x}_s)$和$P(\mathbf{A}^\top \mathbf{x}_t)$的距离能够尽可能地接近。其实这个操作就是迁移成分分析(TCA)。我们仍然使用MMD距离来最小化源域和目标域的最大均值差异。MMD距离是

\begin{equation}
	\left \Vert \frac{1}{n} \sum_{i=1}^{n} \mathbf{A}^\top \mathbf{x}_{i} - \frac{1}{m} \sum_{j=1}^{m} \mathbf{A}^\top \mathbf{x}_{j} \right \Vert ^2_\mathcal{H}
\end{equation}

这个式子实在不好求解。我们引入核方法,化简这个式子,它就变成了

\begin{equation}
	D(\mathcal{D}_s,\mathcal{D}_t)=tr(\mathbf{A}^\top \mathbf{X} \mathbf{M}_0 \mathbf{X}^\top \mathbf{A})
\end{equation}

其中$\mathbf{A}$就是变换矩阵,我们把它加黑加粗,$\mathbf{X}$是源域和目标域合并起来的数据。$\mathbf{M}_0$是一个MMD矩阵:

\begin{equation}
	(\mathbf{M}_0)_{ij}=\begin{cases} \frac{1}{n^2}, & \mathbf{x}_i,\mathbf{x}_j \in \mathcal{D}_s\\ \frac{1}{m^2}, & \mathbf{x}_i,\mathbf{x}_j \in \mathcal{D}_t\\ -\frac{1}{mn}, & \text{otherwise} \end{cases}
\end{equation}

$n,m$分别是源域和目标域样本的个数。

到此为止没有什么创新点,因为这就是一个TCA。

\textit{条件分布适配}

这是我们要做的第二个目标,适配源域和目标域的条件概率分布。也就是说,还是要找一个变换$\mathbf{A}$,使得$P(y_s|\mathbf{A}^\top \mathbf{x}_s)$和$P(y_t|\mathbf{A}^\top \mathbf{x}_t)$的距离也要小。那么简单了,我们再用一遍MMD啊。可是问题来了:我们的目标域里,没有$y_t$,没法求目标域的条件分布!

这条路看来是走不通了。也就是说,直接建模$P(y_t|\mathbf{x}_t)$不行。那么,能不能有别的办法可以逼近这个条件概率?我们可以换个角度,利用类条件概率$P(\mathbf{x}_t|y_t)$。根据贝叶斯公式$P(y_t|\mathbf{x}_t)=p(y_t)p(\mathbf{x}_t|y_t)$,我们如果忽略$P(\mathbf{x}_t)$,那么岂不是就可以用$P(\mathbf{x}_t|y_t)$来近似$P(y_t|\mathbf{x}_t)$?

而这样的近似也不是空穴来风。在统计学上,有一个概念叫做\textit{充分统计量},它是什么意思呢?大概意思就是说,如果样本里有太多的东西未知,样本足够好,我们就能够从中选择一些统计量,近似地代替我们要估计的分布。好了,我们为近似找到了理论依据。

实际怎么做呢?我们依然没有$y_t$。采用的方法是,用$(\mathbf{x}_s,y_s)$来训练一个简单的分类器(比如knn、逻辑斯特回归),到$\mathbf{x}_t$上直接进行预测。总能够得到一些伪标签$\hat{y}_t$。我们根据伪标签来计算,这个问题就可解了。

类与类之间的MMD距离表示为

\begin{equation}
	\sum_{c=1}^{C}\left \Vert \frac{1}{n_c} \sum_{\mathbf{x}_{i} \in \mathcal{D}^{(c)}_s} \mathbf{A}^\top \mathbf{x}_{i} - \frac{1}{m_c} \sum_{\mathbf{x}_{i} \in \mathcal{D}^{(c)}_t} \mathbf{A}^\top \mathbf{x}_{i} \right \Vert ^2_\mathcal{H}
\end{equation}

其中,$n_c,m_c$分别标识源域和目标域中来自第$c$类的样本个数。同样地我们用核方法,得到了下面的式子

\begin{equation}
	\sum_{c=1}^{C}tr(\mathbf{A}^\top \mathbf{X} \mathbf{M}_c \mathbf{X}^\top \mathbf{A})
\end{equation}

其中$\mathbf{M}_c$为

\begin{equation}
	(\mathbf{M}_c)_{ij}=\begin{cases} \frac{1}{n^2_c}, & \mathbf{x}_i,\mathbf{x}_j \in \mathcal{D}^{(c)}_s\\ \frac{1}{m^2_c}, & \mathbf{x}_i,\mathbf{x}_j \in \mathcal{D}^{(c)}_t\\ -\frac{1}{m_c n_c}, & \begin{cases} \mathbf{x}_i \in \mathcal{D}^{(c)}_s ,\mathbf{x}_j \in \mathcal{D}^{(c)}_t \\ \mathbf{x}_i \in \mathcal{D}^{(c)}_t ,\mathbf{x}_j \in \mathcal{D}^{(c)}_s \end{cases}\\ 0, & \text{otherwise}\end{cases}
\end{equation}

现在我们把两个距离结合起来,得到了一个总的优化目标:

\begin{equation}
	\min \sum_{c=0}^{C}tr(\mathbf{A}^\top \mathbf{X} \mathbf{M}_c \mathbf{X}^\top \mathbf{A}) + \lambda \Vert \mathbf{A} \Vert ^2_F
\end{equation}

看到没,通过$c=0 \cdots C$就把两个距离统一起来了!其中的$\lambda \Vert \mathbf{A} \Vert ^2_F$是正则项,使得模型是\textit{良好定义}(Well-defined)的。

我们还缺一个限制条件,不然这个问题无法解。限制条件是什么呢?和TCA一样,变换前后数据的方差要维持不变。怎么求数据的方差呢,还和TCA一样:$\mathbf{A}^\top \mathbf{X} \mathbf{H} \mathbf{X}^\top \mathbf{A} = \mathbf{I}$,其中的$\mathbf{H}$也是中心矩阵,$\mathbf{I}$是单位矩阵。也就是说,我们又添加了一个优化目标是要$\max \mathbf{A}^\top \mathbf{X} \mathbf{H} \mathbf{X}^\top \mathbf{A}$(这一个步骤等价于PCA了)。和原来的优化目标合并,优化目标统一为:

\begin{equation}
	\min \frac{\sum_{c=0}^{C}tr(\mathbf{A}^\top \mathbf{X} \mathbf{M}_c \mathbf{X}^\top \mathbf{A}) + \lambda \Vert \mathbf{A}\Vert^2_F}{ \mathbf{A}^\top \mathbf{X} \mathbf{H} \mathbf{X}^\top \mathbf{A}}
\end{equation}

这个式子实在不好求解。但是,有个东西叫做Rayleigh quotient~\footnote{\url{https://www.wikiwand.com/en/Rayleigh_quotient}},上面两个一样的这种形式。因为$\mathbf{A}$是可以进行拉伸而不改变最终结果的,而如果下面为0的话,整个式子就求不出来值了。所以,我们直接就可以让下面不变,只求上面。所以我们最终的优化问题形式搞成了

\begin{equation}
	\min \quad \sum_{c=0}^{C}tr(\mathbf{A}^\top \mathbf{X} \mathbf{M}_c \mathbf{X}^\top \mathbf{A}) + \lambda \Vert \mathbf{A} \Vert ^2_F \quad \text{s.t.} \quad \mathbf{A}^\top \mathbf{X} \mathbf{H} \mathbf{X}^\top \mathbf{A} = \mathbf{I}
\end{equation}

怎么解?太简单了,可以用拉格朗日法。最后变成了

\begin{equation}
	\left(\mathbf{X} \sum_{c=0}^{C} \mathbf{M}_c \mathbf{X}^\top + \lambda \mathbf{I}\right) \mathbf{A} =\mathbf{X} \mathbf{H} \mathbf{X}^\top \mathbf{A} \Phi 
\end{equation}

其中的$\Phi$是拉格朗日乘子。别看这个东西复杂,又有要求解的$\mathbf{A}$,又有一个新加入的$\Phi$ 。但是它在Matlab里是可以直接解的(用$\mathrm{eigs}$函数即可)。这样我们就得到了变换$\mathbf{A}$,问题解决了。

可是伪标签终究是伪标签啊,肯定精度不高,怎么办?有个东西叫做\textit{迭代},一次不行,我们再做一次。后一次做的时候,我们用上一轮得到的标签来作伪标签。这样的目的是得到越来越好的伪标签,而参与迁移的数据是不会变的。这样往返多次,结果就自然而然好了。

JDA方法是十分经典的迁移学习方法。后续的相关工作通过在JDA的基础上加入额外的损失项,使得迁移学习的效果得到了很大提升。我们在这里简要介绍一些基于JDA的相关工作。

\begin{itemize}
	\item ARTL~(Adaptation Regularization)~\cite{long2014adaptation}: 将JDA嵌入一个结构风险最小化框架中,用表示定理直接学习分类器
	\item VDA~\cite{tahmoresnezhad2016visual}: 在JDA的优化目标中加入了类内距和类间距的计算
	\item \cite{hsiao2016learning}: 在JDA的基础上加入结构不变性控制
	\item \cite{hou2015unsupervised}: 在JDA的基础上加入目标域的选择
	\item JGSA~(Joint Geometrical and Statistical Alignment)~\cite{zhang2017joint}: 在JDA的基础上加入类内距、类间距、标签持久化
	\item JAN~(Joint Adaptation Network)~\cite{long2017deep}: 提出了联合分布度量JMMD,在深度网络中进行联合分布的优化
\end{itemize}

\subsection{动态分布自适应}

\textbf{平衡分布自适应BDA}

在最近的研究中,来自中科院计算所的Wang等人~\cite{wang2017balanced}注意到了JDA的不足:\textit{边缘分布自适应和条件分布自适应并不是同等重要}。回到图~\ref{fig-distribution}表示的两种分布的问题上来。显然,当目标域是图~\ref{fig-distribution-target1}所示的情况时,边缘分布应该被优先考虑;而当目标域是图~\ref{fig-distribution-target2}所示的情况时,条件分布应该被优先考虑。JDA以及后来的扩展工作均忽视了这一问题。

作者提出了BDA方法(Balanced Distribution Adaptation)来解决这一问题。该方法能够根据特定的数据领域,自适应地调整分布适配过程中边缘分布和条件分布的重要性。准确而言,BDA通过采用一种\textit{平衡因子}$\mu$来动态调整两个分布之间的距离
\begin{equation}
\label{equ-mummd}
\begin{split}
DISTANCE(\mathcal{D}_s,\mathcal{D}_t) \approx  (1 &- \mu)DISTANCE(P(\mathbf{x}_s),P(\mathbf{x}_t))\\
&+ \mu DISTANCE(P(y_s|\mathbf{x}_s),P(y_t|\mathbf{x}_t))
\end{split}
\end{equation}
其中$\mu \in [0,1]$表示平衡因子。当$\mu \rightarrow 0$,这表示源域和目标域数据本身存在较大的差异性,因此,边缘分布适配更重要;当$\mu \rightarrow 1$时,这表示源域和目标域数据集有较高的相似性,因此,条件概率分布适配更加重要。综合上面的分析可知,平衡因子可以根据实际数据分布的情况,动态地调节每个分布的重要性,并取得良好的分布适配效果。

其中的平衡因子$\mu$可以通过分别计算两个领域数据的整体和局部的$\mathcal{A}$-distance近似给出。特别地,当$\mu = 0$时,方法退化为TCA;当$\mu = 0.5$时,方法退化为JDA。

我们采用BDA文章中的图来具体地展示出$\mu$的作用。图~\ref{fig-distribution-bda}的结果清晰地显示出,平衡因子可以取得比JDA、TCA更小的MMD距离、更高的精度。

%\begin{figure}[htbp]
%	\centering
%	\includegraphics[scale=0.6]{./figures/fig-distribution-mu.pdf}
%	\caption{平衡因子$\mu$在分布自适应中的作用}
%	\label{fig-distribution-mu}
%\end{figure}

\begin{figure*}[h]
	\centering
	\subfigure[不同方法的MMD距离比较]{
		\includegraphics[scale=0.45]{./figures/fig-distribution-bdammd.pdf}
		\label{fig-distribution-mmd}}

	\subfigure[BDA方法中平衡因子$\mu$的作用]{
		\includegraphics[scale=0.45]{./figures/fig-distribution-mu.pdf}
		\label{fig-distribution-mu}}
	\caption{BDA方法的效果}
	\label{fig-distribution-bda}
\end{figure*}

\textbf{动态分布自适应}

BDA方法是首次给出边缘分布和条件分布的定量估计。然而,其并未解决平衡因子$\mu$的精确计算问题。最近,作者扩展了BDA方法,提出了一个更具普适性的动态迁移框架DDA(Dynamic Distribution Adaptation)~\cite{wang2019transfer}来解决$\mu$值的精确估计问题。

注意到,可以简单地将$\mu$视为一个迁移过程中的参数,通过交叉验证 (cross-validation)来确定其最优的取值$\mu_{opt}$。然而,在本章的无监督迁移学习问题定义中,目标域完全没有标记,故此方式不可行。有另外两种非直接的方式可以对$\mu$值进行估计:随机猜测和最大最小平均法。随机猜测从神经网络随机调参中得到启发,指的是任意从$[0,1]$区间内选择一个$\mu$的值,然后进行动态迁移,其并不算是一种技术严密型的方案。如果重复此过程$t$次,记第$t$次的迁移学习结果为$r_t$,则随机猜测法最终的迁移结果为$r_{rand} = \frac{1}{t} \sum_{i=1}^{t} r_t$。最大最小平均法与随机猜测法相似,可以在$[0,1]$区间内从0开始取$\mu$的值,每次增加0.1,得到一个集合$[0,0.1,\cdots,0.9,1.0]$,然后,与随机猜测法相似,也可以得到其最终迁移结果$r_{maxmin}=\frac{1}{11} \sum_{i=1}^{11} r_i$。

然而,尽管上述两种估计方案有一定的可行性,它们均需要大量的重复计算,给普适计算设备带来了严峻的挑战。另外,上述结果并不具有可解释性,其正确性也无法得到保证。

作者提出的动态迁移方法是首次对$\mu$值进行精确的定量估计方法。该方法利用领域的整体和局部性质来定量计算$\mu$(计算出的值用$\hat{\mu}$来表示)。采用$\mathcal{A}-distance$~\cite{ben2007analysis}作为基本的度量方式。$\mathcal{A}-distance$被定义为建立一个二分类器进行两个不同领域的分类得出的误差。从形式化来看,定义$\epsilon(h)$作为线性分类器$h$区分两个领域$\Omega_s$和$\Omega_t$的误差。则,$\mathcal{A}-distance$可以被定义为:
\begin{equation}
d_A(\Omega_s,\Omega_t) = 2(1 - 2 \epsilon(h)).
\end{equation}

直接根据上式计算边缘分布的$\mathcal{A}-distance$,将其用$d_M$来表示。对于条件分布之间的$\mathcal{A}-distance$,用$d_c$来表示对应于类别$c$的条件分布距离。它可以由式$d_c = d_A(\Omega^{(c)}_s,\Omega^{(c)}_t)$进行计算,其中$\Omega^{(c)}_s$和$\Omega^{(c)}_t$分别表示来自源域和目标域的第$c$个类的样本。最终,$\mu$可以由下式进行计算:
\begin{equation}
\label{eq-meda-mu}
\hat{\mu} = 1 - \frac{d_M}{d_M + \sum_{c=1}^{C} d_c}.
\end{equation}

由于特征的动态和渐近变化性,此估计需要在每一轮迭代中给出。值得注意的是,这是\textbf{首次}给出边缘分布和条件分布的定量估计,对于迁移学习研究具有很大的意义。

具体而言,作者将机器学习问题规约成一个统计机器学习问题,可以用统计机器学习中的结构风险最小化的原则(Structural Risk Minimization, SRM)~\cite{belkin2006manifold,vapnik1998statistical}进行表示学习。在SRM中,分类器$f$可以被表示为:
\begin{equation}
\label{eq-meda-srm}
f = \mathop{\arg\min}_{f \in \mathcal{H}_{K}, (\mathbf{x},y) \sim \Omega_l} J(f(\mathbf{x}),y) + R(f),
\end{equation}
其中第一项表示$f$在有标记数据上的损失,第二项为正则项,$\mathcal{H}_{K}$表示核函数$K(\cdot,\cdot)$构造的希尔伯特空间 (Hilbert space)。符号$\Omega_l$表示有标记的数据领域。在本章的问题中,$\Omega_l = \Omega_s$,即只有源域数据有标记。特别地,由于在迁移学习问题中,源域和目标域数据有着不同的数据分布,为了表示此分布距离,可以进一步将正则项表示成如下的形式:
\begin{equation}
R(f) = \lambda \overline{D_f}(\Omega_s,\Omega_t) + R_f(\Omega_s,\Omega_t),
\end{equation}
其中$\overline{D_f}(\cdot, \cdot)$表示$\Omega_s$和$\Omega_t$的分布距离,$\lambda$为平衡系数,$R_f(\cdot, \cdot)$则为其他形式的正则项。根据公式~(\ref{eq-meda-srm})中的结构风险最小化公式,如果用$g(\cdot)$来表示特征学习过程,则$f$可以被表示为:
\begin{equation}
\label{equ-f-orig}
f = \mathop{\arg\min}_{f \in \sum_{i=1}^{n} \mathcal{H}_{K}} J(f(g(\mathbf{x}_i)),y_i) + \eta ||f||^2_K + \lambda \overline{D_f}(\Omega_s,\Omega_t) + \rho R_f(\Omega_s,\Omega_t),
\end{equation}
其中$||f||^2_K$是$f$的平方标准形式。$\overline{D_f}(\cdot,\cdot)$这一项表示本章提出的动态迁移学习。引入拉普拉斯约束作为$f$的额外正则项~\cite{belkin2006manifold}。$\eta,\lambda$,和$\rho$是对应的正则项系数。

上式则为通用的一个迁移学习框架,可以适用于任何问题。为了对此框架进行学习,作者分别提出了基于流形学习的动态迁移方法MEDA (Manifold Embedded Distribution Alignment)~\cite{wang2018visual}和基于深度学习的动态迁移方法DDAN (Deep Dynamic Adaptation Network)~\cite{wang2019transfer}来进行学习。这两种方法分别如图~\ref{fig-distribution-meda}和~\ref{fig-distribution-ddan}所示。

%\begin{figure}[t!]
%	\centering
%	\begin{subfigure}[1]{0.48\textwidth}
%		\includegraphics[width=\textwidth]{./figures/fig-meda-main.pdf}
%		\caption{流形空间动态迁移MEDA}
%		\label{fig-meda-manifold}
%	\end{subfigure}
%	
%	\hspace{.2in}
%	
%	\begin{subfigure}[1]{0.48\textwidth}
%		\includegraphics[width=\textwidth]{./figures/fig-meda-deep.pdf}
%		\caption{深度网络动态迁移DDAN}
%		\label{fig-meda-deep}
%	\end{subfigure}
%	\caption{基于流形学习和深度学习的动态迁移方法MDDA和DDAN}
%	\label{fig-meda-main}
%\end{figure}

\begin{figure*}[h]
	\centering
	\subfigure[流形空间动态迁移MEDA]{
		\includegraphics[scale=0.45]{./figures/fig-meda-main.pdf}
		\label{fig-distribution-meda}}
	
	\subfigure[深度网络动态迁移DDAN]{
		\includegraphics[scale=0.45]{./figures/fig-meda-deep.pdf}
		\label{fig-distribution-ddan}}
	\caption{动态分布自适应}
	\label{fig-distribution-dda}
\end{figure*}

最近,作者在~\cite{yu2019transfer}中将DDA的概念进一步扩展到了对抗网络中,证明了对抗网络中同样存在边缘分布和条件分布不匹配的问题。作者提出一个动态对抗适配网络DAAN (Dynamic Adversarial Adaptation Networks)来解决对抗网络中的动态分布适配问题,取得了当前的最好效果。图~\ref{fig-daan}展示了DAAN的架构。

\begin{figure}[htbp]
	\centering
	\includegraphics[scale=.35]{./figures/fig-distribution-daan.png}
	\caption{动态对抗适配网络DAAN结构示意图}
	\label{fig-daan}
\end{figure}

\subsection{小结}

综合上述三种概率分布自适应方法,我们可以得出如下的结论:

\begin{enumerate}
	\item 精度比较:DDA > JDA > TCA > 条件分布自适应。
	\item 将不同的概率分布自适应方法用于神经网络,是一个发展趋势。将概率分布适配加入深度网络中,往往会取得比非深度方法更好的结果。
\end{enumerate}


\newpage

\section{第二类方法:特征选择}

特征选择法的基本假设是:源域和目标域中均含有一部分公共的特征,在这部分公共的特征上,源领域和目标领域的数据分布是一致的。因此,此类方法的目标就是,通过机器学习方法,选择出这部分共享的特征,即可依据这些特征构建模型。

图~\ref{fig-feature}形象地表示了特征选择法的主要思路。

\begin{figure}[htbp]
	\centering
	\includegraphics[scale=0.5]{./figures/fig-feature.pdf}
	\caption{特征选择法示意图}
	\label{fig-feature}
\end{figure}

\subsection{核心方法}

这这个领域比较经典的一个方法是发表在2006年的ECML-PKDD会议上,作者提出了一个叫做SCL的方法(Structural Correspondence Learning)~\cite{blitzer2006domain}。这个方法的目标就是我们说的,找到两个领域公共的那些特征。作者将这些公共的特征叫做Pivot feature。找出来这些Pivot feature,就完成了迁移学习的任务。

\begin{figure}[htbp]
	\centering
	\includegraphics[scale=0.3]{./figures/fig-feature-pivot.pdf}
	\caption{特征选择法中的Pivot feature示意图}
	\label{fig-feature-pivot}
\end{figure}

图~\ref{fig-feature-pivot}形象地展示了Pivot feature的含义。Pivot feature指的是在文本分类中,在不同领域中出现频次较高的那些词。


\subsection{扩展}

SCL方法是特征选择方面的经典研究工作。基于SCL,也出现了一些扩展工作。

\begin{itemize}
	\item Joint feature selection and subspace learning~\cite{gu2011joint}:特征选择+子空间学习
	\item TJM (Transfer Joint Matching)~\cite{long2014transfer}: 在优化目标中同时进行边缘分布自适应和源域样本选择
	\item FSSL (Feature Selection and Structure Preservation)~\cite{li2016joint}: 特征选择+信息不变性
\end{itemize}

\subsection{小结}

\begin{itemize}
	\item 特征选择法从源域和目标域中选择提取共享的特征,建立统一模型
	\item 通常与分布自适应方法进行结合
	\item 通常采用稀疏表示$||\mathbf{A}||_{2,1}$实现特征选择
\end{itemize}

\input{chaps/ch08_subspacelearn}
\input{chaps/ch09_deep}
\newpage

\definecolor{dkgreen}{rgb}{0,0.6,0}
\definecolor{gray}{rgb}{0.5,0.5,0.5}
\definecolor{mauve}{rgb}{0.58,0,0.82}

\lstset{ %
	language=Matlab,                % the language of the code
	basicstyle=\footnotesize,           % the size of the fonts that are used for the code
	numbers=left,                   % where to put the line-numbers
	numberstyle=\tiny\color{gray},  % the style that is used for the line-numbers
	stepnumber=2,                   % the step between two line-numbers. If it's 1, each line 
	% will be numbered
	numbersep=5pt,                  % how far the line-numbers are from the code
	backgroundcolor=\color{white},      % choose the background color. You must add \usepackage{color}
	showspaces=false,               % show spaces adding particular underscores
	showstringspaces=false,         % underline spaces within strings
	showtabs=false,                 % show tabs within strings adding particular underscores
	frame=single,                   % adds a frame around the code
	rulecolor=\color{black},        % if not set, the frame-color may be changed on line-breaks within not-black text (e.g. commens (green here))
	tabsize=2,                      % sets default tabsize to 2 spaces
	captionpos=b,                   % sets the caption-position to bottom
	breaklines=true,                % sets automatic line breaking
	breakatwhitespace=false,        % sets if automatic breaks should only happen at whitespace
	title=\lstname,                   % show the filename of files included with \lstinputlisting;
	% also try caption instead of title
	keywordstyle=\color{blue},          % keyword style
	commentstyle=\color{dkgreen},       % comment style
	stringstyle=\color{mauve},         % string literal style
	escapeinside={\%*}{*)},            % if you want to add LaTeX within your code
	morekeywords={*,...}               % if you want to add more keywords to the set
}

\section{上手实践}

以上对迁移学习基本方法的介绍还只是停留在算法的阶段。对于初学者来说,不仅需要掌握基础的算法知识,更重要的是,需要在实验中发现问题。本章的目的是为初学者提供一个迁移学习上手实践的介绍。通过一步步编写代码、下载数据,完成迁移学习任务。在本部分,我们以迁移学习中最为流行的图像分类为实验对象,在流行的Office+Caltech10数据集上完成。

迁移学习方法主要包括:传统的非深度迁移、深度网络的finetune、深度网络自适应、以及深度对抗网络的迁移。教程的目的是抛砖引玉,帮助初学者快速入门。由于网络上已有成型的深度网络的finetune、深度网络自适应、以及深度对抗网络的迁移教程,因此我们不再叙述这些方法,只在这里介绍非深度方法的教程。其他三种方法的地址分别是:

\begin{itemize}
	\item 深度网络的finetune:\href{https://github.com/jindongwang/transferlearning/tree/master/code/deep/finetune_AlexNet_ResNet}{用Pytorch对Alexnet和Resnet进行微调}、\href{https://pytorch.org/tutorials/beginner/transfer_learning_tutorial.html}{使用PyTorch进行finetune}
	\item 深度网络的自适应:\href{https://github.com/jindongwang/transferlearning/tree/master/code/deep/DDC_DeepCoral}{DDC/DCORAL方法的Pytorch代码}
	\item 深度对抗网络迁移:\href{https://github.com/jindongwang/transferlearning/tree/master/code/deep/DANN(RevGrad)}{DANN方法}
\end{itemize}

更多深度迁移方法的代码,请见\url{https://github.com/jindongwang/transferlearning/tree/master/code/deep}。

%\subsection{非深度迁移}

在众多的非深度迁移学习方法中,我们选择最经典的迁移方法之一、发表于IEEE TNN 2011的TCA(Transfer Component Analysis)~\cite{pan2011domain}方法进行实践。为了便于学习,我们同时用Matlab和Python实现了此代码。代码的链接为\url{https://github.com/jindongwang/transferlearning/tree/master/code/traditional/TCA}。下面我们对代码进行简单讲解。

\subsection{TCA方法代码实现}

\subsubsection{Matlab}

\textbf{1. 数据获取}

由于我们要测试非深度方法,因此,选择SURF特征文件作为算法的输入。SURF特征文件可以从网络上~\footnote{\url{https://pan.baidu.com/s/1bp4g7Av}}下载。下载到的文件主要包含4个.mat文件:Caltech.mat, amazon.mat, webcam.mat, dslr.mat。它们恰巧对应4个不同的领域。彼此之间两两一组,就是一个迁移学习任务。每个数据文件包含两个部分:fts为800维的特征,labels为对应的标注。在测试中,我们选择由Caltech.mat作为源域,由amazon.mat作为目标域。Office+Caltech10数据集的介绍可以在本手册的第~\ref{sec-dataset}部分找到。

我们对数据进行加载并做简单的归一化,将最后的数据存入$X_s,Y_s,X_t,Y_t$这四个变量中。这四个变量分别对应源域的特征和标注、以及目标域的特征和标注。代码如下:

\begin{lstlisting}[title=Matlab加载数据, frame=shadowbox]
load('Caltech.mat');     % source domain
fts = fts ./ repmat(sum(fts,2),1,size(fts,2)); 
Xs = zscore(fts,1);    clear fts
Ys = labels;           clear labels

load('amazon.mat');    % target domain
fts = fts ./ repmat(sum(fts,2),1,size(fts,2)); 
Xt = zscore(fts,1);     clear fts
Yt = labels;            clear labels
\end{lstlisting}

\textbf{2. 算法精炼}

TCA主要进行边缘分布自适应。通过整理化简,TCA最终的求解目标是:
\begin{equation}
\label{equ-eigen}
\begin{split}
\left(\mathbf{X} \mathbf{M} \mathbf{X}^\top + \lambda \mathbf{I}\right) \mathbf{A} =\mathbf{X} \mathbf{H} \mathbf{X}^\top \mathbf{A} \Phi 
\end{split}
\end{equation}

上述表达式可以通过Matlab自带的$\mathrm{eigs()}$函数直接求解。$\mathbf{A}$就是我们要求解的变换矩阵。下面我们需要明确各个变量所指代的含义:

\begin{itemize}
	\item $\mathbf{X}$: 由源域和目标域数据共同构成的数据矩阵
	\item $C$: 总的类别个数。在我们的数据集中,$C=10$
	\item $\mathbf{M}_c$: MMD矩阵。当$c=0$时为全MMD矩阵;当$c>1$时对应为每个类别的矩阵。
	\item $\mathbf{I}$:单位矩阵
	\item $\lambda$:平衡参数,直接给出
	\item $\mathbf{H}$: 中心矩阵,直接计算得出
	\item $\Phi$: 拉格朗日因子,不用理会,求解用不到
\end{itemize}

\textbf{3. 编写代码}

我们直接给出精炼后的源码:

\begin{lstlisting}[title=TCA方法的Matlab实现, frame=shadowbox]
function [X_src_new,X_tar_new,A] = TCA(X_src,X_tar,options)
% The is the implementation of Transfer Component Analysis.
% Reference: Sinno Pan et al. Domain Adaptation via Transfer Component Analysis. TNN 2011.

% Inputs: 
%%% X_src          :    source feature matrix, ns * n_feature
%%% X_tar          :    target feature matrix, nt * n_feature
%%% options        :    option struct
%%%%% lambda       :    regularization parameter
%%%%% dim          :    dimensionality after adaptation (dim <= n_feature)
%%%%% kernel_tpye  :    kernel name, choose from 'primal' | 'linear' | 'rbf'
%%%%% gamma        :    bandwidth for rbf kernel, can be missed for other kernels

% Outputs: 
%%% X_src_new      :    transformed source feature matrix, ns * dim
%%% X_tar_new      :    transformed target feature matrix, nt * dim
%%% A              :    adaptation matrix, (ns + nt) * (ns + nt)

%%%%%%%%%%%%%%%%%%%%%%%%%%%%%%%%%%%%%%%%%%%%%%%%%%%%%%%%%

%% Set options
lambda = options.lambda;              
dim = options.dim;                    
kernel_type = options.kernel_type;    
gamma = options.gamma;                

%% Calculate
X = [X_src',X_tar'];
X = X*diag(sparse(1./sqrt(sum(X.^2))));
[m,n] = size(X);
ns = size(X_src,1);
nt = size(X_tar,1);
e = [1/ns*ones(ns,1);-1/nt*ones(nt,1)];
M = e * e';
M = M / norm(M,'fro');
H = eye(n)-1/(n)*ones(n,n);
if strcmp(kernel_type,'primal')
[A,~] = eigs(X*M*X'+lambda*eye(m),X*H*X',dim,'SM');
Z = A' * X;
Z = Z * diag(sparse(1./sqrt(sum(Z.^2))));
X_src_new = Z(:,1:ns)';
X_tar_new = Z(:,ns+1:end)';
else
K = TCA_kernel(kernel_type,X,[],gamma);
[A,~] = eigs(K*M*K'+lambda*eye(n),K*H*K',dim,'SM');
Z = A' * K;
Z = Z*diag(sparse(1./sqrt(sum(Z.^2))));
X_src_new = Z(:,1:ns)';
X_tar_new = Z(:,ns+1:end)';
end
end

% With Fast Computation of the RBF kernel matrix
% To speed up the computation, we exploit a decomposition of the Euclidean distance (norm)
%
% Inputs:
%       ker:    'linear','rbf','sam'
%       X:      data matrix (features * samples)
%       gamma:  bandwidth of the RBF/SAM kernel
% Output:
%       K: kernel matrix
%
% Gustavo Camps-Valls
% 2006(c)
% Jordi (jordi@uv.es), 2007
% 2007-11: if/then -> switch, and fixed RBF kernel
% Modified by Mingsheng Long
% 2013(c)
% Mingsheng Long (longmingsheng@gmail.com), 2013
function K = TCA_kernel(ker,X,X2,gamma)

switch ker
case 'linear'

if isempty(X2)
K = X'*X;
else
K = X'*X2;
end

case 'rbf'

n1sq = sum(X.^2,1);
n1 = size(X,2);

if isempty(X2)
D = (ones(n1,1)*n1sq)' + ones(n1,1)*n1sq -2*X'*X;
else
n2sq = sum(X2.^2,1);
n2 = size(X2,2);
D = (ones(n2,1)*n1sq)' + ones(n1,1)*n2sq -2*X'*X2;
end
K = exp(-gamma*D); 

case 'sam'

if isempty(X2)
D = X'*X;
else
D = X'*X2;
end
K = exp(-gamma*acos(D).^2);

otherwise
error(['Unsupported kernel ' ker])
end
end

\end{lstlisting}

我们将TCA方法包装成函数$\mathrm{TCA}$。注意到TCA是一个无监督迁移方法,不需要接受label作为参数。因此,函数共接受3个输入参数:

\begin{itemize}
	\item $\mathrm{X_{src}}$: 源域的特征,大小为$n_s \times m$
	\item $\mathrm{X_{tar}}$: 目标域的特征,大小为$n_t \times m$
	\item $\mathrm{options}$: 参数结构体,它包含:
	\begin{itemize}
		\item $\lambda$: 平衡参数,可以自由给出
		\item $dim$: 算法最终选择将数据将到多少维
		\item $kernel type$: 选择的核类型,可以选择RBF、线性、或无核
		\item $\gamma$: 如果选择RBF核,那么它的宽度为$\gamma$
	\end{itemize}
\end{itemize}

函数的输出包含3项:
\begin{itemize}
	\item $X_{srcnew}$: TCA后的源域
	\item $X_{tarnew}$: TCA后的目标域
	\item $A$: 最终的变换矩阵
\end{itemize}

\textbf{4. 测试算法}

我们使用如下的代码对TCA算法进行测试:

\begin{lstlisting}
options.gamma = 2;          % the parameter for kernel
options.kernel_type = 'linear';
options.lambda = 1.0;
options.dim = 20;
[X_src_new,X_tar_new,A] = TCA(Xs,Xt,options);

% Use knn to predict the target label
knn_model = fitcknn(X_src_new,Y_src,'NumNeighbors',1);
Y_tar_pseudo = knn_model.predict(X_tar_new);
acc = length(find(Y_tar_pseudo==Y_tar))/length(Y_tar); 
fprintf('Acc=%0.4f\n',acc);
\end{lstlisting}

结果显示如下:
\begin{lstlisting}
	Acc=0.4499
\end{lstlisting}

\subsubsection{Python}

与Matlab代码类似,我们也可以用Python对TCA进行实现,其主要依赖于Numpy和Scipy两个强大的科学计算库。Python版本的TCA代码如下:

\begin{lstlisting}[title=TCA方法的Python实现, frame=shadowbox]

import numpy as np
import scipy.io
import scipy.linalg
import sklearn.metrics
from sklearn.neighbors import KNeighborsClassifier


def kernel(ker, X1, X2, gamma):
K = None
if not ker or ker == 'primal':
K = X1
elif ker == 'linear':
if X2 is not None:
K = sklearn.metrics.pairwise.linear_kernel(np.asarray(X1).T, np.asarray(X2).T)
else:
K = sklearn.metrics.pairwise.linear_kernel(np.asarray(X1).T)
elif ker == 'rbf':
if X2 is not None:
K = sklearn.metrics.pairwise.rbf_kernel(np.asarray(X1).T, np.asarray(X2).T, gamma)
else:
K = sklearn.metrics.pairwise.rbf_kernel(np.asarray(X1).T, None, gamma)
return K


class TCA:
def __init__(self, kernel_type='primal', dim=30, lamb=1, gamma=1):
'''
Init func
:param kernel_type: kernel, values: 'primal' | 'linear' | 'rbf'
:param dim: dimension after transfer
:param lamb: lambda value in equation
:param gamma: kernel bandwidth for rbf kernel
'''
self.kernel_type = kernel_type
self.dim = dim
self.lamb = lamb
self.gamma = gamma

def fit(self, Xs, Xt):
'''
Transform Xs and Xt
:param Xs: ns * n_feature, source feature
:param Xt: nt * n_feature, target feature
:return: Xs_new and Xt_new after TCA
'''
X = np.hstack((Xs.T, Xt.T))
X /= np.linalg.norm(X, axis=0)
m, n = X.shape
ns, nt = len(Xs), len(Xt)
e = np.vstack((1 / ns * np.ones((ns, 1)), -1 / nt * np.ones((nt, 1))))
M = e * e.T
M = M / np.linalg.norm(M, 'fro')
H = np.eye(n) - 1 / n * np.ones((n, n))
K = kernel(self.kernel_type, X, None, gamma=self.gamma)
n_eye = m if self.kernel_type == 'primal' else n
a, b = np.linalg.multi_dot([K, M, K.T]) + self.lamb * np.eye(n_eye), np.linalg.multi_dot([K, H, K.T])
w, V = scipy.linalg.eig(a, b)
ind = np.argsort(w)
A = V[:, ind[:self.dim]]
Z = np.dot(A.T, K)
Z /= np.linalg.norm(Z, axis=0)
Xs_new, Xt_new = Z[:, :ns].T, Z[:, ns:].T
return Xs_new, Xt_new

def fit_predict(self, Xs, Ys, Xt, Yt):
'''
Transform Xs and Xt, then make predictions on target using 1NN
:param Xs: ns * n_feature, source feature
:param Ys: ns * 1, source label
:param Xt: nt * n_feature, target feature
:param Yt: nt * 1, target label
:return: Accuracy and predicted_labels on the target domain
'''
Xs_new, Xt_new = self.fit(Xs, Xt)
clf = KNeighborsClassifier(n_neighbors=1)
clf.fit(Xs_new, Ys.ravel())
y_pred = clf.predict(Xt_new)
acc = sklearn.metrics.accuracy_score(Yt, y_pred)
return acc, y_pred


if __name__ == '__main__':
domains = ['caltech.mat', 'amazon.mat', 'webcam.mat', 'dslr.mat']
for i in [2]:
for j in [3]:
if i != j:
src, tar = 'data/' + domains[i], 'data/' + domains[j]
src_domain, tar_domain = scipy.io.loadmat(src), scipy.io.loadmat(tar)
Xs, Ys, Xt, Yt = src_domain['feas'], src_domain['label'], tar_domain['feas'], tar_domain['label']
tca = TCA(kernel_type='linear', dim=30, lamb=1, gamma=1)
acc, ypre = tca.fit_predict(Xs, Ys, Xt, Yt)
print(acc)

\end{lstlisting}

\textbf{5. 小结}

通过以上过程,我们分别使用Matlab代码和Python代码对经典的TCA方法进行了实验,完成了一个迁移学习任务。其他的非深度迁移学习方法,均可以参考上面的过程。值得庆幸的是,许多论文的作者都公布了他们的文章代码,以方便我们进行接下来的研究。读者可以从Github~\footnote{\url{https://github.com/jindongwang/transferlearning/tree/master/code}}或者相关作者的网站上获取其他许多方法的代码。

\subsection{深度网络的finetune代码实现}

本小节我们用Pytorch实现一个深度网络的finetune。Pytorch是一个较为流行的深度学习工具包,由Facebook进行开发,在Github~\footnote{\url{https://github.com/pytorch/pytorch}}上也进行了开源。

Finetune指的是训练好的深度网络,拿来在新的目标域上进行微调。因此,我们假定读者具有基本的Pytorch知识,直接给出finetune的代码。完整的代码可以在这里~\footnote{\url{https://github.com/jindongwang/transferlearning/tree/master/code/deep/finetune_AlexNet_ResNet}}找到。

我们定义一个叫做finetune的函数,它接受输入的一个已有模型,从目标数据中进行微调,输出最好的模型其结果。其代码如下:

\begin{lstlisting}[title=深度网络的finetune代码实现, frame=shadowbox]

def finetune(model, dataloaders, optimizer):
since = time.time()
best_acc = 0.0
acc_hist = []
criterion = nn.CrossEntropyLoss()
for epoch in range(1, N_EPOCH + 1):
# lr_schedule(optimizer, epoch)
print('Learning rate: {:.8f}'.format(optimizer.param_groups[0]['lr']))
print('Learning rate: {:.8f}'.format(optimizer.param_groups[-1]['lr']))
for phase in ['src', 'val', 'tar']:
if phase == 'src':
model.train()
else:
model.eval()
total_loss, correct = 0, 0
for inputs, labels in dataloaders[phase]:
inputs, labels = inputs.to(DEVICE), labels.to(DEVICE)
optimizer.zero_grad()
with torch.set_grad_enabled(phase == 'src'):
outputs = model(inputs)
loss = criterion(outputs, labels)
preds = torch.max(outputs, 1)[1]
if phase == 'src':
loss.backward()
optimizer.step()
total_loss += loss.item() * inputs.size(0)
correct += torch.sum(preds == labels.data)
epoch_loss = total_loss / len(dataloaders[phase].dataset)
epoch_acc = correct.double() / len(dataloaders[phase].dataset)
acc_hist.append([epoch_loss, epoch_acc])
print('Epoch: [{:02d}/{:02d}]---{}, loss: {:.6f}, acc: {:.4f}'.format(epoch, N_EPOCH, phase, epoch_loss,
epoch_acc))
if phase == 'tar' and epoch_acc > best_acc:
best_acc = epoch_acc
print()
fname = 'finetune_result' + model_name + \
str(LEARNING_RATE) + str(args.source) + \
'-' + str(args.target) + '.csv'
np.savetxt(fname, np.asarray(a=acc_hist, dtype=float), delimiter=',',
fmt='%.4f')
time_pass = time.time() - since
print('Training complete in {:.0f}m {:.0f}s'.format(
time_pass // 60, time_pass % 60))

return model, best_acc, acc_hist

\end{lstlisting}

其中,model可以是由任意深度网络训练好的模型,如Alexnet、Resnet等。

另外,有很多任务也需要用到深度网络来提取深度特征以便进一步处理。我们也进行了实现,代码在\url{https://github.com/jindongwang/transferlearning/blob/master/code/feature_extractor}中。

%
\subsection{深度网络自适应代码}

我们仍然以Pytorch为例,实现深度网络的自适应。具体地说,实现经典的DDC (Deep Domain Confusion)~\cite{tzeng2014deep}方法和与其类似的DCORAL (Deep CORAL)~\cite{sun2016deep}方法。

此网络实现的核心是:如何正确计算DDC中的MMD损失、以及DCORAL中的CORAL损失,并且与神经网络进行集成。此部分对于初学者难免有一些困惑。如何输入源域和目标域、如何进行判断?因此,我们认为此部分应该是深度迁移学习的基础代码,读者应该努力地进行学习和理解。

\textbf{网络结构}

首先我们要定义好网络的架构,其应该是来自于已有的网络结构,如Alexnet和Resnet。但不同的是,由于要进行深度迁移适配,因此,输出层要和finetune一样,和目标的类别数相同。其二,由于要进行距离的计算,我们需要加一个叫做bottleneck的层,用来将最高维的特征进行降维,然后进行距离计算。当然,bottleneck层不加尚可。

我们的网络结构如下所示:

\begin{lstlisting}[title=深度迁移网络代码实现, frame=shadowbox]
import torch.nn as nn
import torchvision
from Coral import CORAL
import mmd
import backbone


class Transfer_Net(nn.Module):
def __init__(self, num_class, base_net='resnet50', transfer_loss='mmd', use_bottleneck=True, bottleneck_width=256, width=1024):
super(Transfer_Net, self).__init__()
self.base_network = backbone.network_dict[base_net]()
self.use_bottleneck = use_bottleneck
self.transfer_loss = transfer_loss
bottleneck_list = [nn.Linear(self.base_network.output_num(
), bottleneck_width), nn.BatchNorm1d(bottleneck_width), nn.ReLU(), nn.Dropout(0.5)]
self.bottleneck_layer = nn.Sequential(*bottleneck_list)
classifier_layer_list = [nn.Linear(self.base_network.output_num(), width), nn.ReLU(), nn.Dropout(0.5),
nn.Linear(width, num_class)]
self.classifier_layer = nn.Sequential(*classifier_layer_list)

self.bottleneck_layer[0].weight.data.normal_(0, 0.005)
self.bottleneck_layer[0].bias.data.fill_(0.1)
for i in range(2):
self.classifier_layer[i * 3].weight.data.normal_(0, 0.01)
self.classifier_layer[i * 3].bias.data.fill_(0.0)

def forward(self, source, target):
source = self.base_network(source)
target = self.base_network(target)
source_clf = self.classifier_layer(source)
if self.use_bottleneck:
source = self.bottleneck_layer(source)
target = self.bottleneck_layer(target)
transfer_loss = self.adapt_loss(source, target, self.transfer_loss)
return source_clf, transfer_loss

def predict(self, x):
features = self.base_network(x)
clf = self.classifier_layer(features)
return clf

\end{lstlisting}

其中Transfer Net是整个网络的模型定义。它接受参数有:

\begin{itemize}
	\item num class: 目标域类别数
	\item base net: 主干网络,例如Resnet等,也可以是自己定义的网络结构
	\item Transfer loss: 迁移的损失,比如MMD和CORAL,也可以是自己定义的损失
	\item use bottleneck: 是否使用bottleneck
	\item bottleneck width: bottleneck的宽度
	\item width: 分类器层的width
\end{itemize}

\textbf{迁移损失定义}

迁移损失是核心。其定义如下:

\begin{lstlisting}[title=深度迁移网络代码实现, frame=shadowbox]

 def adapt_loss(self, X, Y, adapt_loss):
"""Compute adaptation loss, currently we support mmd and coral
Arguments:
X {tensor} -- source matrix
Y {tensor} -- target matrix
adapt_loss {string} -- loss type, 'mmd' or 'coral'. You can add your own loss
Returns:
[tensor] -- adaptation loss tensor
"""
if adapt_loss == 'mmd':
mmd_loss = mmd.MMD_loss()
loss = mmd_loss(X, Y)
elif adapt_loss == 'coral':
loss = CORAL(X, Y)
else:
loss = 0
return loss
\end{lstlisting}

其中的MMD和CORAL是自己实现的两个loss,MMD对应DDC方法,CORAL对应DCORAL方法。其代码在上述github中可以找到,我们不再赘述。

\textbf{训练}

训练时,我们一次输入一个batch的源域和目标域数据。为了方便,我们使用pytorch自带的dataloader。

\begin{lstlisting}[title=深度迁移网络代码实现, frame=shadowbox]
def train(source_loader, target_train_loader, target_test_loader, model, optimizer, CFG):
len_source_loader = len(source_loader)
len_target_loader = len(target_train_loader)
train_loss_clf = utils.AverageMeter()
train_loss_transfer = utils.AverageMeter()
train_loss_total = utils.AverageMeter()
for e in range(CFG['epoch']):
model.train()
iter_source, iter_target = iter(
source_loader), iter(target_train_loader)
n_batch = min(len_source_loader, len_target_loader)
criterion = torch.nn.CrossEntropyLoss()
for i in range(n_batch):
data_source, label_source = iter_source.next()
data_target, _ = iter_target.next()
data_source, label_source = data_source.to(
DEVICE), label_source.to(DEVICE)
data_target = data_target.to(DEVICE)

optimizer.zero_grad()
label_source_pred, transfer_loss = model(data_source, data_target)
clf_loss = criterion(label_source_pred, label_source)
loss = clf_loss + CFG['lambda'] * transfer_loss
loss.backward()
optimizer.step()
train_loss_clf.update(clf_loss.item())
train_loss_transfer.update(transfer_loss.item())
train_loss_total.update(loss.item())
if i % CFG['log_interval'] == 0:
print('Train Epoch: [{}/{} ({:02d}%)], cls_Loss: {:.6f}, transfer_loss: {:.6f}, total_Loss: {:.6f}'.format(
e + 1,
CFG['epoch'],
int(100. * i / n_batch), train_loss_clf.avg, train_loss_transfer.avg, train_loss_total.avg))

# Test
test(model, target_test_loader)
\end{lstlisting}

%
%\subsection{深度对抗网络迁移}
\newpage

\section{迁移学习前沿}

从我们上述介绍的多种迁移学习方法来看,领域自适应(Domain Adaptation)作为迁移学习的重要分类,在近年来已经取得了大量的研究成果。但是,迁移学习仍然是一个活跃的领域,仍然有大量的问题没有被很好地解决。

本节我们简要介绍一些迁移学习领域较新的研究成果。并且,管中窥豹,展望迁移学习未来可能的研究方向。

\subsection{机器智能与人类经验结合迁移}

机器学习的目的是让机器从众多的数据中发掘知识,从而可以指导人的行为。这样看来,似乎“全自动”是我们的终极目标。我们理想中的机器学习系统,似乎就应该完全不依赖于人的干预,靠算法和数据就能完成所有的任务。Google Deepmind公司最新发布的AlphaZero~\cite{silver2017mastering}就实现了这样的愿景:\textit{算法完全不依赖于人提供知识,从零开始掌握围棋知识,最终打败人类围棋冠军。}随着机器学习的发展,似乎人的角色也会越来越不重要。

然而,在目前看来,机器想完全不依赖于人的经验,就必须付出巨大的时间和计算代价。普通人也许根本无法掌握这样的能力。那么,如果在机器智能中,特别是迁移学习的机器智能中,加入人的经验,可以大幅度提高算法的训练水平,这岂不是我们喜闻乐见的?

来自斯坦福大学的研究人员2017年发表在人工智能顶级会议AAAI上的研究成果就率先实践了这一想法~\cite{stewart2017label}。研究人员提出了一种无需人工标注的神经网络,对视频数据进行分析预测。在该成果中,研究人员的目标是用神经网络预测扔出的枕头的下落轨迹。不同于传统的神经网络需要大量标注,该方法完全不使用人工标注。取而代之的是,将人类的知识赋予神经网络。

我们都知道,抛出的物体往往会沿着\textit{抛物线}的轨迹进行运动。这就是研究人员所利用的核心知识。计算机对于这一点并不知情。因此,在网络中,如果加入抛物线这一基本的先验知识,则会极大地促进网络的训练。并且,最终会取得比单纯依赖算法本身更好的效果。

我们认为将机器智能与人类经验结合起来的迁移学习应该是未来的发展方向之一。期待这方面有更多的研究成果发表。

\subsection{传递式迁移学习}

迁移学习的核心是找到两个领域的相似性。这是成功进行迁移的保证。但是,假如我们的领域数据本身就不存在相似性,或者相似性极小,这时候就很容易出现负迁移。负迁移是迁移学习研究中极力需要避免的。

我们由两个领域的相似性推广开来,其实世间万事万物都有一定的联系。表面上看似无关的两个领域,它们也可以由中间的领域构成联系。也就是一种传递式的相似性。例如,领域A和领域B从表面上看,完全不相似。那么,是否可以找到中间的一个领域C,领域C与A和B都有一定的相似性?这样,知识原来不能直接从领域A迁移到领域B,加入C以后,就可以先从A迁移到C,再从C迁移到B。这就是传递迁移学习。

香港科技大学杨强教授的团队率先在2015年数据挖掘顶级会议KDD上提出了这一概念:Transitive transfer learning~\cite{tan2015transitive}。随后,作者又进行了进一步的扩展,将三个领域的迁移,扩展到多个领域。这也是符合我们认知的:原先完全不相似的两个领域,如果它们中间存在若干领域都与这两个领域相似,那么就可以构成一条相似性链条,知识就可以进行链式的迁移。作者提出了远领域的迁移学习(Distant Domain Transfer Learning)~\cite{tan2017distant},用卷积神经网络解决了这一问题。

在远领域迁移学习中,作者做出了看起来并不符合常人认知的实验:由人脸图片训练好的分类器,迁移识别飞机图像。研究人员采用了人脸和飞机中间的一系列类别,例如头像、头盔、水壶、交通工具等。在实验中,算法自动地选择相似的领域进行迁移。结果表明,在初始迁移阶段,算法选择的大多是与源领域较为相似的类别;随着迁移的进行,算法会越来越倾向于选择与目标领域相似的类别。这也是符合我们的基本认知的。最终的对比实验表明,这种远领域的迁移学习比直接训练分类器的精度会有极大的提升。图~\ref{fig-future-ddtl}展示了知识迁移的过程。

\begin{figure}[htbp]
	\centering
	\includegraphics[scale=0.35]{./figures/fig-future-ddtl.pdf}
	\caption{远领域迁移学习示意图}
	\label{fig-future-ddtl}
\end{figure}

传递迁移学习目前的研究成果还十分稀少。我们期待这一领域会有更多好的成果出现。

\subsection{终身迁移学习}

我们在进行迁移学习时,往往不知道应该选择怎么样的算法。通常都通过人为地不断尝试来确定要用的方法。这个过程无疑是浪费时间,而且充满了不确定性。已有的迁移学习方法,基本都是这样。我们在拿到一个新问题时,如何选择迁移学习算法达到最好的效果?从人的学习过程中来说,人总是可以从以前的经验中学习知识。那么,既然我们已经实验了很多次迁移学习算法,我们能不能让机器也可以从我们这些实验中选择知识?这是符合我们人类认知的:不可能每遇到一个新问题,我们就从头开始做吧?

同样是来自香港科技大学杨强教授团队就开始了这一方面的研究工作。他们提出一种学习迁移(L2T, Learning to Transfer)的框架~\cite{wei2017learning},解决\textit{何时迁移、要迁移什么、怎么迁移}的问题。方法分为两个部分:从已有的迁移学习方法和结果中学习迁移的经验,然后再把这些学习到的经验应用到新来的数据。

首先要明确学习目标。跟以前的迁移学习方法都有所不同,以往的方法都是要学习最好的迁移函数,而这个问题的目标是要使得方法尽可能地具有泛化能力。因此,它的学习目标是:\textbf{以往的经验}!对的,就是经验。这也是符合我们人类认知的。我们一般都是知道的越多,这个人越厉害(当然,《权利的游戏》里的Jon Snow除外,因为他什么也不知道)。因此,这个方法的目标就是要尽可能多地从以往的迁移知识中学习经验,使之对于后来的问题具有最好的泛化能力。 

那么\textit{什么是迁移的经验}?这个在文中叫做transfer learning experience。作者这样定义:$Ee=(S_e,T_e,a_e,l_e)$。其中,$S_e,T_e$分别是源域和目标域,这个我们都知道。$a_e$表示一个迁移学习算法,这个算法有个下标叫$e$,表示它是第$e$种算法。与之对应,选择了这种算法,它对不迁移情况下的表现有个提升效果,这个效果就叫做$le$。

总结一下,什么叫迁移学习的经验?就是说,在一对迁移任务中,我选择了哪种算法后,这种算法对于我任务效果有多少提升。这个东西,就叫做迁移!这和我们人类学习也是具有相似性的:人们常说,失败是成功之母,爱迪生说,我实验了2000多种材料做灯泡都是失败的,但是我最起码知道了这2000多种材料不适合做灯泡!这就是人类的经验!我们人类就是从跌倒中爬起,从失败中总结教训,然后不断进步。学习算法也可以! 

后面的过程就是,综合性地学习这个迁移过程,使得算法根据以往的经验,得到一个特征变换矩阵$\mathbf{W}$。学习到了这个变换矩阵以后,下一步的工作就是要把学习到的东西应用于新来的数据。如何针对新来的数据进行迁移?我们本能地要利用刚刚学习到的这个$\mathbf{W}$。但是不要忘了,这个变换矩阵只是对旧的那些经验学习到的,对新的数据可能效果不好,不能直接用。怎么办?我们要更新它!

这里要注意的是,针对新来的数据,我们的这个变换矩阵应该是有所改变的:这也对,数据变了,当然变换矩阵就要变。那么,如何更新?作者在这里提出的方法是,新的矩阵应该是能在新的数据上表现效果最好的那个。这时,我们的问题就完成了。 

图~\ref{fig-future-l2t}简要表示了该算法的学习过程。

\begin{figure}[htbp]
	\centering
	\includegraphics[scale=0.35]{./figures/fig-future-l2t.pdf}
	\caption{终身迁移学习示意图}
	\label{fig-future-l2t}
\end{figure}

这有点类似增量学习的工作:模型针对新数据不断更新优化。这部分的研究才刚刚开始。

\subsection{在线迁移学习}

我们都知道迁移学习可以用来解决训练数据缺失的问题,很多迁移学习方法都获得了长足的进步。给定一个要学习的目标域数据,我们可以用已知标签的源域数据来给这个目标域数据构造一个分类器。但是这些方法都存在很大的一个问题:它们都是采用离线方式(offline)进行的。什么是离线方式?就是说,一开始,源域和目标域数据都是给出来的,我们直接做完迁移,这个过程就结束了。We are done。

但是真实的应用往往不是这样的:数据往往是一点一点源源不断送来的。也就是说,我们一开始的时候,也许只有源域数据,目标域数据要一点一点才能过来。这就是所谓的“在线迁移学习”。这个概念脱胎于“在线学习”的模式,在线学习是机器学习中一个重要的研究概念。

就目前来说,在线迁移学习方面的工作较少。第一篇在线迁移学习的工作由新加坡管理大学的Steven Hoi发表在2010年的机器学习顶级会议ICML上。作者提出了OTL框架~\cite{zhao2010otl},可以对同构和异构数据很好地进行迁移学习。

近年来,研究者发表了一些在线迁移学习相关的文章。其中包括在多个源域和目标域上的OTL~\cite{wu2017online,yan2017online},在线特征选择迁移变换~\cite{wang2014online,zhang2017online},在线样本集成迁移~\cite{gao2012online,patilknowledge}等。

特别地,\cite{jaini2016online}提出了用贝叶斯的方法学习在线的HMM迁移学习模型,并应用于行为识别、睡眠监测,以及未来流量分析。这是一篇有代表性的应用工作。

总结来看,目前在在线迁移学习方面的研究工作总体较少,发展空间巨大。在可以预见的未来,我们期待更多的研究者可以从事这一领域的研究。将深度网络、对抗学习结合入在线迁移学习,使得这一领域的发展越来越好。

\subsection{迁移强化学习}

Google公司的AlphaGo系列在围棋方面的成就让\textit{强化学习}这一术语变得炙手可热。用深度神经网络来进行强化学习也理所当然地成为了研究热点之一。不同于传统的机器学习需要大量的标签才可以训练学习模型,强化学习采用的是\textit{边获得样例边学习}的方式。特定的反馈函数决定了算法的最优决策。

深度强化学习同时也面临着重大的挑战:没有足够的训练数据。在这个方面,迁移学习却可以利用其他数据上训练好的模型帮助训练。尽管迁移学习已经被应用于强化学习~\cite{taylor2009transfer},但是它的发展空间仍然还很大。强化学习在自动驾驶、机器人、路径规划等领域正发挥着越来越重要的作用。我们期待在未来有更多的研究成果可以问世。

\subsection{迁移学习的可解释性}

深度学习取得众多突破性成果的同时,其面临的可解释性不强却始终是一个挑战。现有的深度学习方法还停留在"黑盒子"阶段,无法产生足够有说服力的解释。同样的,迁移学习也有这个问题。即使世间万物都有联系,它们更深层次的关系也尚未得到探索。领域之间的相似性也正如同海森堡"测不准原理"一般无法给出有效的结论。为什么领域A和领域B更相似,而和领域C较不相似?目前也只是停留在经验阶段,缺乏有效的理论证明。

另外,迁移学习算法也存在着可解释性弱的问题。现有的算法均只是完成了一个迁移学习任务。但是在学习过程中,知识是如何进行迁移的,这一点还有待进一步的实验和理论验证。最近,澳大利亚悉尼大学的研究者们发表在国际人工智能联合会IJCAI 2017上的研究成果有助于理解特征是如何迁移的~\cite{liu2017understanding}。

用深度网络来做迁移学习,其可解释性同样有待探索。最近,Google Brain的研究者们提出了神经网络的"核磁共振"现象~\footnote{\url{https://github.com/tensorflow/lucid}},对神经网络的可解释性进行了有趣的探索。


\input{chaps/ch12_conclusion}
\newpage
\section{附录}

\subsection{迁移学习相关的期刊和会议}

迁移学习仍然是一个蓬勃发展的研究领域。最近几年,在顶级期刊和会议上,越来越多的研究者开始发表文章,不断提出迁移学习的新方法,不断开拓迁移学习的新应用领域。

在这里,我们对迁移学习相关的国际期刊和会议作一小结,以方便初学者寻找合适的论文。这些期刊会议可以在表~\ref{tb-appendix-journalconference}中找到。

\begin{table}[htbp]
	\centering
	\caption{迁移学习相关的期刊和会议}
	\label{tb-appendix-journalconference}
	\resizebox{1\textwidth}{!}{
	\begin{tabular}{|c|c|c|c|}
		\hline
		\textbf{序号} & \textbf{简称} & \textbf{全称} & \textbf{领域} \\ \hline \hline
		\multicolumn{4}{|c|}{国际期刊} \\ \hline
		1 & JMLR & Journal of Machine Learning Research & 机器学习 \\ \hline
		2 & MLJ & Machine Learning Journal & 机器学习 \\ \hline
		3 & AIJ & Artificial Intelligence Journal & 人工智能 \\ \hline
		4 & TKDE & IEEE Transactions on Knowledge and Data Engineering & 数据挖掘 \\ \hline
		5 & TIST & ACM Transactions on Intelligent Science and Technology & 数据挖掘 \\ \hline
		6 & PAMI & IEEE Transactions on Pattern Analysis and Machine Intelligence & 计算机视觉 \\ \hline
		7 & IJCV & International Journal of Computer Vision & 计算机视觉 \\ \hline
		8 & TIP & IEEE Transactions on Image Processing & 计算机视觉 \\ \hline
		9 & PR & Pattern Recognition & 模式识别 \\ \hline
		10 & PRL & Pattern Recognition Letters & 模式识别 \\ \hline \hline
		\multicolumn{4}{|c|}{国际会议} \\ \hline
		1 & ICML & International Conference on Machine Learning & 机器学习 \\ \hline
		2 & NIPS & Annual Conference on Neural Information Processing System & 机器学习 \\ \hline
		3 & IJCAI & International Joint Conference on Artificial Intelligence & 人工智能 \\ \hline
		4 & AAAI & AAAI conference on Artificial Intelligence & 人工智能 \\ \hline
		5 & KDD & \begin{tabular}[c]{@{}c@{}}ACM SIGKDD\\   Conference on Knowledge Discovery and Data Mining\end{tabular} & 数据挖掘 \\ \hline
		6 & ICDM & IEEE International Conference on Data Mining & 数据挖掘 \\ \hline
		7 & CVPR & \begin{tabular}[c]{@{}c@{}}IEEE Conference on Computer Vision and\\   Pattern Recognition\end{tabular} & 计算机视觉 \\ \hline
		8 & ICCV & IEEE International Conference on Computer Vision & 计算机视觉 \\ \hline
		9 & ECCV & European Conference on Computer Vision & 计算机视觉 \\ \hline
		10 & WWW & International World Wide Web Conferences & 文本、互联网 \\ \hline
		11 & CIKM & International Conference on Information and Knowledge Management & 文本分析 \\ \hline
		12 & ACMMM & ACM International Conference on Multimedia & 多媒体 \\ \hline
	\end{tabular}
}
\end{table}

\subsection{迁移学习研究学者}

这里列出了一些迁移学习领域代表性学者以及他们的最具代表性的工作,以供分享。

一般这些工作都是由他们一作,或者是由自己的学生做出来的。当然,这里所列的文章比起这些大牛发过的文章会少得多,这里仅仅列出他们最知名的工作。本部分开源在了Github~\footnote{\url{https://github.com/jindongwang/transferlearning/blob/master/doc/scholar_TL.md}},会一直有更新,欢迎补充!

\textbf{应用研究}

\textbf{1. \href{http://www.cs.ust.hk/~qyang/}{Qiang Yang} @ HKUST}

迁移学习领域权威大牛。他所在的课题组基本都做迁移学习方面的研究。迁移学习综述《A survey on transfer learning》就出自杨强老师课题组。他的学生们:

\textbf{1). \href{http://www.cs.ust.hk/~qyang/}{Sinno J. Pan} @ NTU}

现为老师,详细介绍见第二条。

\textbf{2). \href{https://sites.google.com/view/btan/home}{Ben Tan}}

主要研究传递迁移学习(transitive transfer learning),现在腾讯做高级研究员。代表文章:

Transitive Transfer Learning. KDD 2015.

Distant Domain Transfer Learning. AAAI 2017.

\textbf{3).  \href{https://scholar.google.com/citations?user=Ks81aO0AAAAJ&hl=zh-CN&oi=ao}{Derek Hao Hu}}

主要研究迁移学习与行为识别结合,目前在Snap公司。代表文章:

Transfer Learning for Activity Recognition via Sensor Mapping. IJCAI 2011.

Cross-domain activity recognition via transfer learning. PMC 2011.

Bridging domains using world wide knowledge for transfer learning. TKDE 2010.

\textbf{4). \href{https://sites.google.com/site/vincentwzheng/}{Vencent Wencheng Zheng}}

也做行为识别与迁移学习的结合,目前在新加坡一个研究所当研究科学家。代表文章:

User-dependent Aspect Model for Collaborative Activity Recognition. IJCAI 2011.

Transfer Learning by Reusing Structured Knowledge. AI Magazine.

Transferring Multi-device Localization Models using Latent Multi-task Learning. AAAI 2008.

Transferring Localization Models Over Time. AAAI 2008.

Cross-Domain Activity Recognition. Ubicomp 2009.

Collaborative Location and Activity Recommendations with GPS History Data. WWW 2010.

\textbf{5). Ying Wei}

做迁移学习与数据挖掘相关的研究。代表工作:

Instilling Social to Physical: Co-Regularized Heterogeneous Transfer Learning. AAAI 2016.

Transfer Knowledge between Cities. KDD 2016.

Learning to Transfer. arXiv 2017.

其他还有很多学生都做迁移学习方面的研究,更多请参考杨强老师主页。

\textbf{2. \href{http://www.cs.ust.hk/~qyang/}{Sinno J. Pan} @ NTU}

杨强老师学生,比较著名的工作是TCA方法。现在在NTU当老师,一直都在做迁移学习研究。代表工作:

A Survey On Transfer Learning. TKDE 2010. [最著名的综述]

Domain Adaptation via Transfer Component Analysis. TNNLS 2011. [著名的TCA方法]

Cross-domain sentiment classification via spectral feature alignment. WWW 2010. [著名的SFA方法]

Transferring Localization Models across Space. AAAI 2008.


\textbf{3. \href{http://www.lxduan.info/}{Lixin Duan} @ UESTC}

毕业于NTU,现在在UESTC当老师。代表工作:

Domain Transfer Multiple Kernel Learning. PAMI 2012.

Visual Event Recognition in Videos by Learning from Web Data. PAMI 2012.


\textbf{4. \href{http://ise.thss.tsinghua.edu.cn/~mlong/}{Mingsheng Long} @ THU}

毕业于清华大学,现在在清华大学当老师,一直在做迁移学习方面的工作。代表工作:

Dual Transfer Learning. SDM 2012.

Transfer Feature Learning with Joint Distribution Adaptation. ICCV 2013.

Transfer Joint Matching for Unsupervised Domain Adaptation. CVPR 2014.

Learning transferable features with deep adaptation networks. ICML 2015. [著名的DAN方法]

Deep Transfer Learning with Joint Adaptation Networks. ICML 2017.


\textbf{5. \href{http://people.eecs.berkeley.edu/~jhoffman/}{Judy Hoffman} @ UC Berkeley \& Stanford}

Feifei Li的博士后,现在当老师。她有个学生叫做Eric Tzeng,做深度迁移学习。代表工作:

Simultaneous Deep Transfer Across Domains and Tasks. ICCV 2015.

Deep Domain Confusion: Maximizing for Domain Invariance. arXiv 2014.

Adversarial Discriminative Domain Adaptation. CVPR 2017.


\textbf{6. \href{http://www.intsci.ac.cn/users/zhuangfuzhen/}{Fuzhen Zhuang} @ ICT, CAS}

中科院计算所当老师,主要做迁移学习与文本结合的研究。代表工作:

Transfer Learning from Multiple Source Domains via Consensus Regularization. CIKM 2008.


\textbf{7. \href{https://www.cs.cornell.edu/~kilian/}{Kilian Q. Weinberger} @ Cornell U.}

现在康奈尔大学当老师。\href{http://www.cse.wustl.edu/~mchen/}{Minmin Chen}是他的学生。代表工作:

Distance metric learning for large margin nearest neighbor classification. JMLR 2009.

Feature hashing for large scale multitask learning. ICML 2009.

An introduction to nonlinear dimensionality reduction by maximum variance unfolding. AAAI 2006. [著名的MVU方法]

Co-training for domain adaptation. NIPS 2011. [著名的Co-training方法]

\textbf{8. \href{http://www.cse.wustl.edu/~mchen/}{Fei Sha} @ USC}

USC教授。他曾经的学生\href{http://www.cecs.ucf.edu/faculty/boqing-gong/}{Boqing Gong}提出了著名的GFK方法。代表工作:

Connecting the Dots with Landmarks: Discriminatively Learning Domain-Invariant Features for Unsupervised Domain Adaptation. ICML 2013.

Geodesic flow kernel for unsupervised domain adaptation. CVPR 2012. [著名的GFK方法]


\textbf{9. Mahsa Baktashmotlagh @ U. Queensland}

现在当老师。主要做流形学习与domain adaptation结合。代表工作:

Unsupervised Domain Adaptation by Domain Invariant Projection. ICCV 2013.

Domain Adaptation on the Statistical Manifold. CVPR 2014.

Distribution-Matching Embedding for Visual Domain Adaptation. JMLR 2016.


\textbf{10. \href{https://www.microsoft.com/en-us/research/people/baochens/}{Baochen Sun} @ Microsoft}

现在在微软。著名的CoRAL系列方法的作者。代表工作:

Return of Frustratingly Easy Domain Adaptation. AAAI 2016.

Deep coral: Correlation alignment for deep domain adaptation. ECCV 2016.


\textbf{11. Wenyuan Dai}

著名的第四范式创始人,虽然不做研究了,但是当年求学时几篇迁移学习文章至今都很高引。代表工作:

Boosting for transfer learning. ICML 2007. [著名的TrAdaboost方法]

Self-taught clustering. ICML 2008.


\textbf{理论研究}

\textbf{1. \href{http://www.gatsby.ucl.ac.uk/~gretton/}{Arthur Gretton} @ UCL}

主要做two-sample test。代表工作:

A Kernel Two-Sample Test. JMLR 2013.

Optimal kernel choice for large-scale two-sample tests. NIPS 2012. [著名的MK-MMD]


\textbf{2. \href{https://cs.uwaterloo.ca/~shai/}{Shai Ben-David} @ U.Waterloo}

很多迁移学习的理论工作由他给出。代表工作:

Analysis of representations for domain adaptation. NIPS 2007.

A theory of learning from different domains. Machine Learning 2010.


\textbf{3. \href{https://alex.smola.org/}{Alex Smola} @ CMU}

做一些机器学习的理论工作,和上面两位合作比较多。代表工作非常多,不列了。

\textbf{4. \href{https://alex.smola.org/}{John Blitzer} @ Google}

著名的SCL方法提出者,现在也在做机器学习。代表工作:

Domain adaptation with structural correspondence learning. ECML 2007. [著名的SCL方法]


\textbf{5. \href{http://www.iro.umontreal.ca/~bengioy/yoshua_en/index.html}{Yoshua Bengio} @ U.Montreal}

深度学习领军人物,主要做深度迁移学习的一些理论工作。代表工作:

Deep Learning of Representations for Unsupervised and Transfer Learning. ICML 2012.

How transferable are features in deep neural networks? NIPS 2014.

Unsupervised and Transfer Learning Challenge: a Deep Learning Approach. ICML 2012.


\textbf{6. \href{http://www.iro.umontreal.ca/~bengioy/yoshua_en/index.html}{Geoffrey Hinton} @ U.Toronto}

深度学习领军人物,也做深度迁移学习的理论工作。

Distilling the knowledge in a neural network. NIPS 2014.

\subsection{迁移学习资源汇总}

\begin{itemize}
	\item (可能是有史以来)最全的迁移学习资料库,(文章/资料/代码/数据):	\url{https://github.com/jindongwang/transferlearning}
	\item 迁移学习视频教程:	\url{https://www.youtube.com/watch?v=qD6iD4TFsdQ}
	\item 知乎专栏“机器有颗玻璃心”中《小王爱迁移》系列:	\url{https://zhuanlan.zhihu.com/p/27336930},用浅显易懂的语言深入讲解经典+最新的迁移学习文章
	\item 迁移学习与领域自适应论文分享与笔记	Paperweekly:\url{http://www.paperweekly.site/collections/231/papers}
	\item 迁移学习与领域自适应公开数据集
	\url{https://github.com/jindongwang/transferlearning/blob/master/doc/dataset.md}
\end{itemize}

\subsection{迁移学习常用算法及数据资源}
\label{sec-dataset}

为了研究及测试算法性能,收集了若干来自图像、文本、及时间序列的公开数据集。并且,为了与自己所研究的算法进行对比,收集了领域内若干较前沿的算法及其相关代码,以便后期进行算法的比较测试。

\textbf{数据集:}

表~\ref{tb-dataset}收集了迁移学习领域常用的数据集。这些数据集的详细介绍和下载地址,在Github~\footnote{\url{https://github.com/jindongwang/transferlearning/blob/master/data/dataset.md}}上可以找到。我们还在Benchmark~\footnote{\url{https://github.com/jindongwang/transferlearning/blob/master/data/benchmark.md}}上提供了一些常用算法的实验结果。

\begin{table}[htbp]
	\centering
	\caption{迁移学习相关的图像、文本、和时间序列数据集统计信息}
	\label{tb-dataset}
	\begin{tabular}{|c|c|c|c|c|c|}
		\hline
		\textbf{序号} & \textbf{数据集} & \textbf{类型} & \textbf{样本数} & \textbf{特征数} & \textbf{类别数} \\ \hline \hline
		1 & USPS & 字符识别 & 1800 & 256 & 10 \\ \hline
		2 & MNIST & 字符识别 & 2000 & 256 & 10 \\ \hline
		3 & PIE & 人脸识别 & 11554 & 1024 & 68 \\ \hline
		4 & COIL20 & 对象识别 & 1440 & 1024 & 20 \\ \hline
		5 & Office+Caltech & 对象识别 & 2533 & 800 & 10 \\ \hline
		6 & ImageNet & 图像分类 & 7341 & 4096 & 5 \\ \hline
		7 & VOC2007 & 图像分类 & 3376 & 4096 & 5 \\ \hline
		8 & LabelMe & 图像分类 & 2656 & 4096 & 5 \\ \hline
		9 & SUN09 & 图像分类 & 3282 & 4096 & 5 \\ \hline
		10 & Caltech101 & 图像分类 & 1415 & 4096 & 5 \\ \hline
		11 & 20newsgroup & 文本分类 & 25804 & / & 6 \\ \hline
		12 & Reuters-21578 & 文本分类 & 4771 & / & 3 \\ \hline
		13 & OPPORTUNITY & 行为识别 & 701366 & 27 & 4 \\ \hline
		14 & DSADS & 行为识别 & 2844868 & 27 & 19 \\ \hline
		15 & PAMAP2 & 行为识别 & 1140000 & 27 & 18 \\ \hline
	\end{tabular}
\end{table}

\textbf{各数据集的简要介绍:}

\textbf{1. 手写体识别图像数据集}

MNIST和USPS是两个通用的手写体识别数据集,它们被广泛地应用于机器学习算法评测的各个方面。USPS数据集包括7,291张训练图片和2,007张测试图片,图片大小为16$\times$16。MNIST数据集包括60,000张训练图片和10,000张测试图片,图片大小28$\times$28。USPS和MNIST数据集分别服从显著不同的概率分布,两个数据集都包含10个类别,每个类别是1-10之间的某个字符。为了构造迁移学习人物,在USPS中随机选取1,800张图片作为辅助数据、在MNIST中随机选取2,000张图片作为目标数据。交换辅助领域和目标领域可以得到另一个分类任务MNIST vs USPS。图片预处理包括:将所有图片大小线性缩放为16$\times$16,每幅图片用256维的特征向量表征,编码了图片的像素灰度值信息。辅助领域和目标领域共享特征空间和类别空间,但数据分布显著不同。

\textbf{2. 人脸识别图像数据集}

PIE代表“朝向、光照、表情”的英文单词首字母,该数据集是人脸识别的基准测试集,包括68个不同人物的41,368幅人脸照片,图片大小为32$\times$32,每个人物的照片由13个同步的相机(不同朝向)、21个不同曝光程度拍摄。简单起见,实验中采用PIE的预处理集,包括2个不同子集PIE1和PIE2,是从正面朝向的人脸照片集合(C27)中按照不同的光照和曝光条件随机选出。按如下方法构造分类任务PIEI vs PIE2:将PIE1作为辅助领域、PIE2作为目标领域;交换辅助领域和目标领域可以得到分类任务PIE2 vs PIEI。这样,辅助领域和目标领域分别由不同光照、曝光条件的人脸照片组成,从而服从显著不同的概率分布。

\textbf{3. 对象识别数据集}

COIL20包含20个对象类别共1,440张图片;每个对象类别包括72张图片,每张图片拍摄时对象水平旋转5度(共360度)。每幅图片大小为32$\times$32,表征为1,024维的向量。实验中将该数据集划分为两个不相交的子集COIL1和COIL2:COIL1包括位于拍摄角度为$[0\textdegree,85\textdegree]\cup[180\textdegree,265\textdegree]$(第一、三象限)的所有图片;COIL2包括位于拍摄角度为$[90\textdegree,175\textdegree]\cup[270\textdegree,355\textdegree]$(第二、四象限)的所有图片。这样,子集COIL1和COIL2的图片因为拍摄角度不同而服从不同的概率分布。将COIL1作为辅助领域、COIL2作为目标领域,可以构造跨领域分类任务COIL1 vs COIL2;交换辅助领域和目标领域,可以得到另外一个分类任务COIL2 vs COIL1。

Office是视觉迁移学习的主流基准数据集,包含3个对象领域Amazon(在线电商图片)、Webcam(网络摄像头拍摄的低解析度图片)、DSLR(单反相机拍摄的高解析度图片),共有4,652张图片31个类别标签。Caltech-256是对象识别的基准数据集,包括1个对象领域Caltech,共有30,607张图片256个类别标签。对每张图片抽取SURF特征,并向量化为800维的直方图表征,所有直方图向量都进行减均值除方差的归一化处理,直方图码表由K均值聚类算法在Amazon子集上生成。具体共有4个领域C(Caltech-256), A(Amazon), W(Webcam)和D(DSLR),从中随机选取2个不同的领域作为辅助领域和目标领域,则可构造$4 \times 3 = 12$个跨领域视觉对象识别任务,如$A \rightarrow D, A \rightarrow C, \cdots, C \rightarrow W$。

\textbf{4. 大规模图像分类数据集}

大规模图像分类数据集包含了来自5个域的图像数据:ImageNet、VOC 2007、SUN、LabelMe、以及Caltech。它们包含5个类别的图像数据:鸟,猫,椅子,狗,人。对于每个域的数据,均使用DeCaf~\cite{donahue2014decaf}进行特征提取,并取第6层的特征作为实验使用,简称DeCaf6特征。每个样本有4096个维度。

\textbf{5. 通用文本分类数据集}

20-Newsgroups数据集包含约20,000个文档,4个大类分别为comp, rec,sci和talk,每个大类包含4个子类,详细信息如表2.2所示。在实验中构造了6组跨领域二分类任务,每组任务由4个大类中随机选取2个大类构成,一个大类记为正例,另一个大类记为负例,6个任务组具体为comp vs rec, comp vs sci, comp vs talk,  rec vs sci,  rec vs talk和、sci vs talk。每个跨领域分类任务(包括辅助领域和目标领域)按如下方法生成:每个任务组p VS的两个大类p和Q分别包含4个子类P1、P2、P3、P4和Q1、Q2、Q3、Q4;随机选取p的两个子类(如P1、P2)与Q的两个子类(如Q1、Q2)构成辅助领域,其余子类(P的P3和P4和Q3和Q4构成目标领域。以上构造策略既保证辅助领域和目标领域是相关的,因为它们都来自同样的大类;又保证辅助领域和目标领域是不同的,因为它们来自不同的子类。每个任务组P VS Q可以生成36个分类任务,总计6个任务组共生成6$\times$36 = 216个分类任务。数据集经过文本预处理后包含25,804个词项特征和15,033个文档,每个文档由tf-idf向量表征。

Reuters-21578是一个较难的文本数据集,包含多个大类和子类。其中最大3个大类为orgs, people和place,可构造6个跨领域文本分类任务orgs vs people,people vs orgs, orgs vs place, place vs orgs, people vs place和place vs people。

\textbf{6. 行为识别公开数据集}

行为识别是典型的时间序列分类任务。为了测试算法在时间序列任务上的性能,收集了3个公开的行为识别数据集:OPPORTUNITY、DASDS和PAMAP2。OPPORTUNITY数据集包含4个用户在智能家居中的多种不同层次的行为。DAADS数据集包含8个人的19种日常行为。PAMAP2数据集包含9个人的18种日常生活行为。所有数据集均包括加速度计、陀螺仪和磁力计三种运动传感器。

\textbf{算法:}

收集的数据表征、迁移学习等相关领域的基准算法如表~\ref{tb-existing_algo}所示。我们还在持续更新的Github~\footnote{\url{https://github.com/jindongwang/transferlearning}}上提供了各种算法的实现代码。

\begin{table}[htbp]
	\centering
	\caption{收集的公开算法}
	\label{tb-existing_algo}
	\resizebox{1\textwidth}{!}{
	\begin{tabular}{|c|c|c|c|}
		\hline
		\textbf{序号} & \textbf{算法简称} & \textbf{算法全称} & \multicolumn{1}{c|}{\textbf{出处}} \\ \hline \hline
		1 & PCA & principal component analysis & \cite{fodor2002survey}  \\ \hline
		2 & KPCA & kernel principal component analysis & \cite{fodor2002survey} \\ \hline
		3 & TCA & transfer component analysis & \cite{pan2011domain} \\ \hline
		4 & GFK & geodesic flow kernel & \cite{gong2012geodesic} \\ \hline
		5 & TKL & transfer kernel learning & \cite{long2015domain} \\ \hline
		6 & TSL & transfer subspace learning & \cite{si2010bregman} \\ \hline
		7 & JDA & joint distribution adaptation & \cite{long2013transfer} \\ \hline
		8 & TJM & transfer joint matching & \cite{long2014transfer} \\ \hline
		9 & DAN & deep adaptation network & \cite{long2015learning} \\ \hline
		10 & JAN & joint adaptation network & \cite{long2017deep} \\ \hline
		11 & DTMKL & domain transfer multiple kernel learning & \cite{duan2012domain} \\ \hline
		12 & JGSA & joint geometrical and statistical adaptation & \cite{zhang2017joint} \\ \hline
		13 & SCA & scatter component analysis & \cite{ghifary2017scatter} \\ \hline
		14 & ARTL & adaptation regularization & \cite{long2014adaptation} \\ \hline
		15 & TrAdaBoost & transfer learning based adaboost & \cite{dai2007boosting} \\ \hline
		16 & GNMF & graph regularized NMF & \cite{cai2011graph} \\ \hline
		17 & CORAL & Correlation Alignment & \cite{sun2016return} \\ \hline
		18 & SDA & Subspace Distribution Alignment & \cite{sun2015subspace} \\ \hline
	\end{tabular}
}
\end{table}


%----------------------------------SYSTEM DESIGN------------------------------------------


% -----------------------------------REFERENCE----------------------------------------


\bibliographystyle{apalike}
\bibliography{refs}

\end{document}

%%% Local Variables:
%%% mode: xelatex
%%% TeX-master: t
%%% End:
